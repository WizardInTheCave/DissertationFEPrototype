\documentclass{article}
%\documentclass[conference]{IEEEtran}

\setcounter{secnumdepth}{0}
%[a4paper, 11pt]
\usepackage{comment} % enables the use of multi-line comments (\ifx \fi) 
\usepackage{fullpage} % changes the margin

\usepackage{graphicx}
%\usepackage{lscape}
%\usepackage{rotating}
\usepackage{pdflscape}
\usepackage{wrapfig}
\usepackage{gensymb}
\usepackage{graphicx} 
\usepackage[utf8]{inputenc}
\usepackage{siunitx}
\usepackage{amssymb}
\usepackage{amsmath}
\usepackage{listings}
\usepackage{float}

%\usepackage[titletoc]{appendix}


\renewcommand{\labelenumii}{\theenumii}
\renewcommand{\theenumii}{\theenumi.\arabic{enumii}.}

\usepackage{enumitem}
\usepackage{subcaption}
\usepackage{algorithm2e}
\usepackage[noend]{algpseudocode}
%\usepackage{pgffor}
\usepackage[T1]{fontenc}
\usepackage{xcolor}

\newcommand{\CMwidth}{-1.7cm}
\newcommand{\CMheight}{2.7cm}

\usepackage{hyperref}

\hypersetup{
	colorlinks=true,
	linkcolor=black,
	citecolor=black,
	filecolor=blue,
	menucolor=black,
	urlcolor=blue
}


\setcounter{secnumdepth}{5}% Include \subsubsection in ToC

\def\changemargin#1#2{\list{}{\rightmargin#2\leftmargin#1}\item[]}
\let\endchangemargin=\endlist 


\begin{document}
%Header-Make sure you update this information!!!!
%\noindent
%\large\textbf{Post/Pre-Lab X Report} \hfill \textbf{FirstName LastName} \\
%\normalsize ECE 100-003 \hfill Teammates: Student1, Student2 \\
%Prof. Oruklu \hfill Lab Date: XX/XX/XX \\
%TA: Adam Sumner \hfill Due Date: XX/XX/XX

%\title{Dissertation: Hybrid Methods for Finite Element Meshing}
%\author{Jack Bradbrook (psyjb4)}
%\date{April 5, 2017}
%\maketitle

\begin{center}

\noindent

\begin{figure}[h!!]
  \centerline{\includegraphics[width=90mm, scale=0.5]{../Graphics/nottinghamLogo.jpeg}}
\end{figure}


\vspace{0.8cm}
\LARGE
\textbf{Hybrid Methods for Finite Element Meshing}\\
\normalsize
\vspace{0.8cm}
Submitted April 2017, in partial fulfilment of
the conditions of the award of the degree BSc (Hons) Computer Science.\\
\vspace{0.4cm}
\Large
\textbf{Jack Bradbrook}\\
\vspace{0.4cm}
\textbf{psyjb4}\\\\
\normalsize
\vspace{0.4cm}
School of Computer Science\\
\vspace{0.4cm}
University Of Nottingham\\ 
\vspace{0.4cm}

\noindent
I hereby declare that this dissertation is all my own work, except as indicated
in the text: \\ 

\vspace{0.4cm}
\noindent
Signature: \\ 

\vspace{2.0cm}
\noindent
Date: 24/04/2017 \\ 

\vspace{0.4cm}

\end{center}

%   abstract-text
%
\newpage
\renewcommand{\abstractname}{\LARGE Abstract} 
\begin{abstract}
\vspace{0.6cm}
%\section{Abstract}

\noindent
\textbf{Background to Finite Element Analysis (FEA)} \\ 
In simple terms Finite Element Analysis (FEA) is a technique that takes a structure and divides it into a number of small segments known as elements, the collection of elements form a mesh or more precisely a Finite Element Mesh (FEM). FEA then calculates a parameter (in this case stress) on each of the elements in the FEM from which it’s possible to identify where failures are most likely to occur. To calculate high stress accurately and in as little time as possible it’s necessary to create additional smaller elements in the areas most likely to experience high stress, this process is known as refinement. Traditionally those elements initially experiencing high stress are chosen for refinement then the stress calculations are repeated on the new FEM and process is repeated until the values of stress in each element cease to increase. \\


\noindent
\textbf{Objective of the project} \\ 
The objective of the project was to develop a new approach to refining a Finite Element Mesh (FEM) which  combines the traditional stress based approach, described above, with a machine learning approach based on a set of established engineering rules. The success of the project will be determined by whether the new hybrid approach delivers the same accuracy as the traditional approach but also by whether it’s able to achieve this accuracy in less time than the traditional approach. \\ 


\noindent
\textbf{The Approach} \\ 
Experienced engineers understand where high levels of stress and failures are most likely to occur in a structure, for example along edges where two surfaces are joined together. This information can be described in a set of user defined edge specifications. The proposed approach was to undertake the FEM refinement process in two stages, first applying the traditional stress based approach to a mesh and then further refining elements based on edge specifications. The result is a new mesh with additional elements created in areas where high stresses are most likely to occur. Having generated a new FEM using the combined approach it’s necessary to check the quality of each element to ensure its shape has not become distorted before ordering the data to allow re-calculation of stresses. Finally, the stress data is output in a format for analysis and presentation after all the planned iterations of refinement are complete. \\

\noindent
\textbf{The Solution} \\
The project used commercially available software (LISA) to calculate the stress in each element of the FEM. All other functions were implemented using bespoke software written as part of the project. In summary the bespoke software performed the following functions:

\begin{itemize}
\item Applied edge rules to a mesh to identify elements requiring refinement.

\item Performed all mesh refinement (for both approaches)

\item Arranged newly formed elements into an order acceptable to LISA.

\item Performed quality checks on refined elements e.g. how distorted they were. 

\item Compared the results of various combinations of approaches through graphs 

\end{itemize}
\noindent
The solution was built using a modular and open architecture to allow for uncertainties during development and to accommodate future enhancements; all code was written using C\#. \\ 


\noindent
\textbf{Evaluation of Findings} \\ 
The solution was used to evaluate the effects of applying forces to three separate models, the analysis identified where stresses appeared on each model and how quickly they could be identified using each of the following approaches:
\begin{itemize}
\item FEA using a stress based refinement.

\item FEA using a heuristic (rule) based refinement. 

\item FEA after using a hybrid approach i.e. a combination of stress and heuristic (rule) based refinement.
\end{itemize}

\noindent
For each approach the project collected data on the following parameters:

\begin{itemize}

\item The time it took to undertake a defined number iterations of mesh refinement.

\item The quality of the elements (how distorted they were).

\item The number of elements generated by the approach.

\item The level of stress revealed by the approach.

\end{itemize}
\noindent
The collected data was plotted to allow an easy comparison of each of the approaches. \\


\noindent
\textbf{Conclusion} \\
From the analysis undertaken it can be concluded that the hybrid approach located high stress points as accurately as the traditional stress based approach, these findings where consistent across all three models used. However, it was not possible to demonstrate conclusively whether the hybrid approach achieved the required results in less time than the stress based approach, this in part was due to having limited time available to test the approach on more complex models where run times are measured in hours rather than the tens of seconds which were the times taken for the most complex model used in this project. The developed system worked well and provides a valuable research tool for further work using more complex models of the type found in industry. The system allows further development of edge specifications without the need for additional coding and the ability to add additional refinement strategies with minimal effort as a consequence of a well defined data model and programming interface. 

\end{abstract}

\newpage

\tableofcontents

%\listoffigures

\newpage

%\section{Introduction}

The overall aim of this project is to perform the task of refining a Finite Element Mesh (FEM) using a stress based method as is typical of FEM refinement in conjunction with an approach that uses techniques from Artificial Intelligence and Machine Learning. It should be possible to reason about the quality of a mesh produced using both methods analytically in order to evaluate the success of the approach and potential for continued use.\\

\noindent
Finite Element Analysis (FEA) is a method widely used across different engineering domains to simulate structures under certain conditions. The method works by taking a geometry defined in continuous terms, discretize it into a mesh system before calculating the property values for each of the discrete elements. This allows engineers to observe the effect the conditions have on the entire structure, see figure 1. \\ 

\noindent
The success of the project will be determined by implementation of a system that is able to combine the two approaches described above in order to refine a mesh to of a quality comparable to that of a successful stress based method.


\section{Introduction}

The overall aim of this project is to perform the task of refining a Finite Element Mesh (FEM), (see section 2.1 for description of FEM) using a hybrid approach which combines stress based method as is traditionally used for FEM refinement with an approach that uses techniques from Artificial Intelligence and Machine Learning. \\

\noindent
Finite Element Analysis (FEA) is a method widely used across different engineering domains to simulate structures under certain conditions. A typical step within the design process for mechanical components involves first designing the component within a CAD application before running a simulation on the design using knowledge of the conditions it is expected to perform under. Having defined the components geometry in continuous terms using CAD software a Finite Element System will break it down into a number of segments (known as elements) which together form a mesh (known as a Finite Element Mesh (FEM). The values of stress within  each of the discrete elements is calculated and then the process of breaking the mesh down into further smaller elements in areas where high levels of stress are encountered, the process is then repeated in an iterative manner until the levels of stress cease to increase or after a defined number of iterations. FEA allows engineers to observe the effect the conditions have on the entire structure, see figure 1. \\ 

\noindent
The success of the project will be determined by implementation of a system that is able to combine the stress and AI based refinement methods in order to refine a mesh of a quality comparable to that of a successful stress based method. Additionally there is benefit in  achieving this quality in less time than the traditional stress based approach. \\ 

\noindent
This Dissertation starts with a review of the FEA process followed by looking at some of approaches that have been taken to refine and evaluate the quality of meshes automatically before continuing to describe the design and discuss challenges of implementing the system. The system is then demonstrated as capable of correctly performing this task in the final evaluation section. \\

\section{Motivation and Background}
During a year long industrial placement at a major aerospace company I worked closely with mechanical engineers, who routinely ran computer programs to calculate the stresses through physical components used in aeronautical designs. The programs used Finite Element Analysis (FEA) to accurately calculate the stresses across the components. However, applying FEA to detailed structures of considerable size can take many hours in execution time to obtain a result. I discovered that it was not uncommon for experienced engineers to correctly predict the regions of high stress revealed by the analysis prior to its completion. This led me to consider how the knowledge of an experienced engineer could be used by an FEA process to reduce the time required to calculate the stress for mechanical components and was consequently the inspiration for the project. \\ 

\noindent
Over the past forty years FEA has emerged as a prominent technology for simulating complex real world engineering problems \cite{cite0, DolsakPaper94}. FEA works by solving a system of differential equations with each equation representing a single element in a geometric mesh. By doing this FEA is able to generate highly accurate approximations for the properties of complex physical systems \cite{DolsakPaper94, IntroductionToFE}. The method can also be highly computationally expensive with the complexity typically increasing exponentially with the model size \cite{DolsakPaper94}. Analysis therefore proves to be highly time consuming and costly for individuals and organisations to conduct \cite{ConsultRuleIntellSystemFE, cite03}.\\


\subsection{Properties of Finite Element Models}
Although this is a computer science dissertation rather than a mechanical engineering dissertation it is still important to briefly outline the general principals and properties that underpin the FEA method in order to have a general appreciation for how the design and evaluation of the final software system was conducted. \\ 

\noindent
Finite Element Models have several key properties that need to be specified by the engineers who create them, the configuration of these properties greatly determine the results obtained from the models execution. The first of these properties that an FE model has is the mesh (known as a Finite Element Mesh or FEM). The mesh is constructed out of nodes, points which act as intersections between the second component- elements which are either a polygon or a polyhedron between the nodes. Nodes and elements are important concepts as they provide the theoretical framework for reasoning about the other properties of a mesh and hence the overall quality of the model \cite{IntroductionToFE}.\\ 

\noindent
Elements within FE models can be different types each defined by their shape. Different shaped elements are selected based on the type of structure that is being assessed and the simulation conditions, some typical examples would be a quadrilateral also known as a quad4 and a triangle (tri3). Appendix B shows the different element types supported by the FE solver (used to calculate stresses in elements) used for this project which is called LISA. \\ 

\noindent
In every type of analysis that the FE method is used for (thermal, structural, fluid flow, electrical) there is a specific differential equation associated with each of the elements. In order to achieve overall convergence of the model the equations must be solved simultaneously to achieve a value for each of the discrete elements \cite{IntroductionToFE}. For this project attention is given specifically to the problem of FEA meshing in the context of static structural analysis where the value calculated across each element is its stress. Stress is defined as force over a given area in an object and arises from some external force being applied. Static structural analysis is one of the most common engineering applications of FEA and has a large  body of research that is relevant to this project. \cite{DolsakPaper94}\cite{IntroductionToFE}.\\

\noindent
In addition to the nodes and elements FE models also contain loads and constraints. Loads can be thought of as the phenomena from the outside world that is acting on the structure and consequently inducing some kind of physical effect on it. Loads are used by engineers to model the conditions which the structure will be expected to perform under when it is manufactured and enters operation. \\

\noindent
Constraints are another fundamental concept that describe where the model is attached to the outside world. When computing stresses through the model there needs to be an area through which the stresses are assumed to leave. FEA is only able to calculate the stresses through the model using the law of conservation of energy i.e. energy cannot be created or destroyed meaning that any energy entering the system as an input such as force needs a means by which to leave it, the constraint. \\

\noindent
For example in figure 6 showing a suspension bridge model, the simulation is to be run with the forces induced upon the cables and the towers in the negative x direction as represented by the green vectors. The corresponding constraint area through which the force must leave is specified as the base of each bridge pillar and represented by multiple red arrows on each corresponding corner.\\

\noindent
The final piece of information needed in order to calculate the stresses through the model is material data. Material data is associated with the models elements and is usually defined using two main parameters which are:

\begin{itemize}
\item Youngs Modulus - The ratio of stress over strain for a given material, i.e. for an amount of internal force endured by a material how much does it deform, a material such as rubber therefore has a low value for Young's modulus while diamond has a high value \cite{YoungsModulus}.

\item Poissons ratio - Amount of deformation that occurs perpendicular to the force that is applied to the material \cite{PossionsRatio}. 
\end{itemize}

\noindent
For the sake of simplicity all structures used to evaluate the final software solution have assumed steel as their material property. Steel is a common material used within manufacturing of many mechanical components and does not exhibit any abnormal properties. This is beneficial for evaluation as it removes variability in the results that could arise from selecting a more complex material. The system I have designed would also work with these more complex materials however the process of assessing the results could potentially be much harder without a more extensive engineering background and better knowledge of the specific materials. 

\subsection{Limitations and general considerations}
\noindent
An important consideration when conducting FEA is the trade-off of a model's accuracy against the time it takes for the solver to perform an analysis of the model. A major variable determining this trade-off is the model's mesh structure which discretizes the problem so that a simulation can be run on it. A mesh that is finer is more computationally expensive but also produces results of greater accuracy. It is therefore desirable to generate a mesh which is fine where accuracy is most needed but coarse where it is not \cite{cite04}. \\


\noindent
For engineers the value obtained through computing the stresses under a particular set of conditions is feedback on the quality of their design. Ideally the results from an analysis will provide a good understanding of where the design is weak and how concentrated this weakness is. This information is used to either help verify the designs' quality or alternatively inform changes to its geometry or material properties so as to reduce stress on subsequent analysis \cite{cite06}.\\

\noindent
To understand the gradient of stress within a part of the model the mesh needs to be designed carefully. As each element can only display values calculated from its edge nodes a smooth gradient requires a higher concentration of elements in areas under higher stress. A high quality mesh is therefore considered to have a higher concentration of elements in areas of predicted high stress while retaining lower concentration elsewhere, thereby revealing weaknesses in the design while minimising the models runtime.\\

\noindent
Traditionally the automated mesh refinement process consists of computing stresses for a model with an initial coarse mesh and low computation cost, once rough stresses have been computed the elements in areas of higher stress can be divided recursively into additional elements in order to achieve smoother gradients on further executions \cite{cite03}. Figure 1 shows a mesh which has been refined in an area of higher stress thus providing a clearer indication of a components weakness. \\ 

\noindent
Unfortunately for many large models this method for refining a mesh is still excessively costly \cite{DolsakPaper91}. It is therefore worth investigating use of alternative approaches posed by the field of computer science and artificial intelligence that could support the traditional high stress meshing approach. \\ \\ \\ \\

\begin{figure}[H]
  \centerline{\includegraphics[width=150mm, scale=0.5]{../Graphics/StressedCorner.png}}
  \caption{Mesh refinement in corner under high stress image source: (\cite{HighStressCorner})}
  \label{fig:boat1}
\end{figure}
\section{Related Work}
Many approaches have so far been taken in an attempt to improve a computer's ability to perform the task of finite element meshing. The following subsections present an overview of the research conducted for the various aspects of the project which are more explicitly outlined under ``Aims and Objectives"

\subsection{Traditional Subdivision Approaches}
Multiple approaches exist for subdividing different types of meshing with the most common being hierarchical refinementent also known as h-refinement and relocation refinement or r-refinement\cite{HandPRefinements} \cite{RRefinement}\\ 

\noindent
\textbf{h-refinement: }
H-refinement is the process of recursively refining a mesh by splitting elements into additional sub elements. This process can be performed for elements with both both triangular and quadrilateral shapes \cite{HandPRefinements}. \\ 

\noindent
\textbf{r-refinement: }
R-refinement is a method which attempts to improve the quality of the stress gradient without the alteration of the mesh element count and thus the computational cost. This is achieved by the relocation of elements within the mesh which effectively increases the size of elements in areas of low stress, while reducing the size in areas where stress is high \cite{RRefinement}.


\subsection{Stress Refinement}
Refining a mesh based on some model parameter is consistent across all types of FE modelling. Execution of a model with a generic mesh is first required so as to obtain a set of results by which further refinement can be targeted. Using unique ids for nodes allows results from the previous iteration to be compared against the mesh and refinement to be focused on those nodes exhibiting a high amount of a specified property \cite{FiniteElementMeshRefinement}.


%A number of methods exist to perform mesh refinement based on a stress gradient for a mesh,


\subsection{Uses of Artificial Intelligence and Machine Learning}

\noindent
Within the domains of AI and machine learning methods such as neural networks \cite{NeuralNetworks}, case based reasoning \cite{caseBasedReasoning} and inductive logic programming \cite{DolsakPaper94} have been adopted to facilitate generation of meshes comparable to that of human experts.  Similarly there has also been effort to combine multiple numerical methods simultaneously for solving re meshing problems \cite{TraditionalHybridRefinement} although effort to combine the two approaches does not appear to have so far been made.\\ 

\noindent
Due to the difficulty of obtaining meshes which hold commercial interest the majority of researchers working in this field have had to resort to the use of training sets developed within academia \cite{DolsakPaper91}. The primary issue associated with this is that often the training data does not exhibit the level of complexity that you would expect in many industrial sectors. Many researchers must accept this as a limitation or agree commercial terms with an organisation in order to gain access to their models \cite{DittmerMeshQualityMet}.\\ 

\noindent
Having reviewed a variety of different AI based applications of FE the use of Inductive Logic Programming (ILP) used by Bojan Dolsak et al is of greatest interest. ILP is a machine learning method first presented by Stephen Muggleton in his 1991 paper ``Inductive Logic Programming" \cite{MuggletonILP}. Muggleton suggests that the traditional approaches of machine learning which rely on use of extensive data sets and statistical analysis are poor in the case of many real world problems for which data is not available \cite{ILPYoutubeLecture}. Muggleton cites Human learning as an example of use of ILP style techniques where understanding of new concepts is achieved not through crunching large volumes of data points but instead  the use of induction on a relatively concise set of background facts and examples obtained from previous life experiances \cite{ILPYoutubeLecture}. \\ 

\noindent
ILP uses three types of input information in order to hypothesise additional facts about the system. These three types of input information are: \\ 

\begin{itemize}
\item Positive examples  of what constitutes an area that is well meshed
\item Negative example of areas that are poorly meshed
\item Background facts
\end{itemize}

\noindent
Using this information ILP is capable of hypothesising rules by determining which rules can exist within the system where given the set of background facts all positive examples are satisfied while few or none of the negative examples are. Although ILP requires a body of additional metadata associated with each mesh this is easier to obtain making ILP a highly practical solution. Along with his publication Muggletons also released his implementation of an ILP algorithm as a program titled ``Golem" \cite{Golem}, Golem was applied by Dolsak to the problem of mesh refinement with a training set of just five meshes \cite{DolsakPaper94}. The resulting rule set when applied to subsequent models was able to correctly classify and re mesh areas with an average accuracy of 78\% for a range of geometries \cite{DolsakPaper94} \cite{appOfILPToFEMeshDesign}. \\

\noindent
Dolsak's choice of metadata for the ILP method to generate mesh rules is based on the classification of edges within the FE model. Dolsak recognises that edges act as an important intersections within the model and as such provide useful items of reference when designing heruristics with which to reason about the model \cite{DolsakPaper94} \cite{appOfILPToFEMeshDesign}.  For example if it is know that an edge has a force applied close to it based on the initial model conditions then other edges that intersect it should be additionally refined \cite{DolsakPaper91} \cite{appOfILPToFEMeshDesign}. \\ 

\noindent
The format of the rules also make them attractive for experimenting with as part of a hybrid method since the method determines how to refine the mesh based on the arrangement of edges. This detail of analysis of the mesh is at a comparable detail to that of a traditional splitting method such as h-refinement which is likely to improve the ease with which the two methods can be combined simultaneously in the latter stages of the project. \\ 

\subsection{Quality metrics}
\noindent
Finally work has also been done on establishing valid metrics for assessing the quality of a mesh automatically \cite{DittmerMeshQualityMet, NeuralNetworks} Metrics for meshes have been researched far more extensively than AI methods due to their use for comparing different stress based refinements. There are also cases of common metrics being used for industrial meshing applications \cite{DittmerMeshQualityMet}. Although there are metrics for assessing a mesh on a global level such as element count score the consensus is that due to the variation in meshes this is less reliable than assessing quality based on the properties of individual elements within the mesh \cite{DittmerMeshQualityMet}. \\

\noindent
Localised metrics associated with the quality of each element have shown to be accurate for predicting the overall quality of a mesh when taking the average for each metric across all elements \cite{DittmerMeshQualityMet}. The quality of an elements shape is important since the stress values which are computed for the area within the element are calculated using the stress values at each of the nodes which enclose it \cite{IntroductionToFE}. Elements are typically deformed near parts of the geometry where its shape simply does not allow a uniform element to be placed, an example of where this has occurred can be seen in figure 2.\\ \\ \\ \\ \\ \\ \\ \\ \\ \\ 

\begin{figure}[!h]
  \centerline{\includegraphics[width=120mm, scale=1]{../Graphics/MeshQualityDeterioration.png}}
  \caption{Example of how elements can be distorted in order to fit a geometry which will result in deterioration of gradient quality (image source: \cite{PoorFEElementShapes})}
\end{figure}

\noindent
Some key shape metrics identified by Dittmer et al include (ideal values are for elements of quadrilateral type as used in  the current prototypes):

\begin{enumerate}[label=\Alph*]
\item Aspect ratio – longest side / shortest side, ideal value is 1
\item Maximum corner angle - widest internal angle to element, ideal is 90\degree
\item Maximum parallel deviation - how skewed the element is, ideal is 0\degree
\end{enumerate}

\section{Description of the work}
\subsection{Aims and Objectives}
The aim of this project was to design, build and analyse a system for refining a mesh by combining a method from the domain of AI or machine learning with a stress based refinement method. The desirable end result will be a hybrid method of meshing which produces reasonably good solution accuracy whilst requiring less user intervention and reduced computational cost.\\

\noindent
The project can be broken into thee main sections of research and implementation which can each be considered a high level objective:\\ 

\begin{enumerate}[label=\Alph*]

\item The first objective is to research and develop both a traditional stress refinement and an AI re meshing algorithm developed by industry or by academia. These algorithms should be able to run independently on a set of example models.

\item Secondly a process will need to be devised for combining the two re meshing methods to varying degrees will be required. This will make it possible to test the effects of a hybrid method across the same set of models. Through this it can be established whether there is notable benefit to combining the different approaches and if so to what degree is it effective.

\item The third objective will be to use justifiable metrics for assessing the quality of a given mesh. This will allow for a much greater range of hybrid variations to be tried without requiring inspection from an expert. 
\end{enumerate}

\subsection{System specification}
To demonstrate success in achieving the objectives of the project it is important to have tractability from the requirements through to the solution and lastly verification and validation. This section describes the current systems \textit{functional requirements} (what the system will do) and \textit{non-functional} requirements (its constraints) based upon evaluation of the research I have conducted in conjunction with discussions with the project supervisor: Dr Jason Atkin. Functional requirements have primarily been listed under their respective high level subsystems that are responsible for encapsulating their functionality. \\ 

\noindent
Although it has not been developed as part of the project the application responsible for solving the finite element models has been included as part of the systems requirements since it highly influences the overall scope of the project and much of the design associated with other subsystems which were developed for the project. \\ 

\subsection{Functional Requirements}

\begin{enumerate}
\item FE integration: System will be able to interface with a third party finite element application

\begin{enumerate}
\item The finite element applications solver must be able to solve a mesh based on its model configuration.
\item The finite element applications solver must be able to execute a model programmatically
\item The finite element applications solver must be able to output stress data at different points on the mesh.
\item The finite element application will provide a graphical representation of the model.
\item It will be possible to manipulate the model that the finite element application uses programmatically.
\item It should be possible to manipulate the model that the finite element application uses from within its graphical user interface.
\end{enumerate}

\item Mesh refinement: System will be able to perform different kinds of finite element mesh refinement

\begin{enumerate}
\item The system will be able to refine a finite element mesh using a stress based refinement method
\item The system will be able to refine a finite element mesh using a non-stress based refinement method

\item A non stress based refinement method will adapt the mesh using background information about mesh design which has been previously trained.

\item The system will be able to combine the two methods to produce a coherent mesh which the FE application is able to successfully solve in order to obtain results for stress and displacement.
\item The system will be able to combine both methods to varying degrees will be performed automatically by the system without direct user intervention.
\item The system will re mesh using both stress and non-stress based refinement using quadrilateral elements.
\item System will adapt weighting associated with each method based upon the metrics computer for the mesh in the systems previous iteration.
\end{enumerate}

\item Quality assessment: System will provide the operator with results about the quality of meshes based on metrics obtained from research.

\begin{enumerate}
\item An assessment will be conducted automatically for every mesh iteration that occurs.
\item System will assess quality based on a variety of metrics to ensure overall robustness of measurement. 
\item The metrics will be computed for both individual elements within the model and for the entire mesh.
\end{enumerate}
\end{enumerate}

\subsection{Non-Functional Requirements}


\textbf{Design:} The system architecture will be developed using the object oriented design principals of SOLID to allow for clear interfaces between the different functional components. Functional programming practices will be adopted through use of the .NET Language Integrated Query or LINQ framework. This will help to simplify the code and improve reliability. Where functions and classes are written their length will be kept to a minimum to reduce complexity and allow for reuse wherever possible. \\ 

\noindent
\textbf{Documentation:} The system will be comprehensively documented at both a code level and at an architecture level. At a code level C\# doc comments will be written to provide a comprehensive summary of each function. This will allow the tool Doxygen \cite{Doxygen} to generate a full set of developer documentation upon completion of the software implementation which will be included as an appendix. Doc comments will also help to encourage small functions. \\

\noindent
\textbf{General applicability:} In order to demonstrate that hybrid methods are a feasible means of approaching meshing problems the resulting software should be able to successfully execute on a range of models with varying geometry. The range of geometries should be representative of typical structural variation.
\section{System Design}

\subsection{System Overview}
Determining the overall design of the system was initially hard since it was not clear exactly how many subsystems would be needed to mesh, evaluate and interface with LISA, what was clear was that the system would essentially be performing an optimisation procedure and as such needed to be driven iteratively towards a goal. The variable complexity and uncertainty surrounding the different parts of the project meant ensuring the architecture remained modular with well defined interfaces allowing components to easily be added or modified as the project progressed. \\ 

\begin{figure}[!h]
  \centerline{\includegraphics[width=150mm, scale=1]{../Graphics/SystemDesignDiagram.jpeg}}
  \caption{High level design of the system with its different modules}
  \label{fig:h-refinementImp}
\end{figure}


\subsection{Modular Architecture}
The modular architecture was crucial for allowing meshing algorithms and quality metrics to be replaced as necessary. At best the quality of the output could could only be predicted for each method before it was integrated into the system and executed in a range of different scenarios. To have tightly coupled these individual components would have rendered the overall system a failure in the event that any one of them failed. Instead the loose coupling of the architecture has enabled the system to be considered as more of a framework for testing the effects of combining different meshing approaches in order to generate a hybrid method.\\

\noindent
Although the system was highly modular It was also still desirable to maintain an architecture hierarchy so that classes could be developed independently but easily integrated. Composition was therefore generally favoured over inheritance as a means of building the architecture. Static classes and methods were also used when needing to write utility functions that were required by multiple high level subsystems and therefore did no fit especially well into any particular one. Examples of these are generic vector algebra operations such as dot product, matrix determinant and calculating surface normals.

\noindent
At the highest level namespaces were used to break down the class groups appropriately, namespaces also naturally structured as folders within the Visual Studio (VS) solution explorer (see appendix D) which made navigating the project and finding components much easier as the system expanded in size.

%\subsection{Mesh improvement Loop}
%As with many optimisation problems the refinement process is driven iteratively through a loop. Within the main loop all %interfacing with LISA, Mesh refinement and analysis is conducted which results in an updated version of the model that can be %handed to the subsequent iteration.
\subsection{Third Party FE Application}
In order to demonstrate the potential feasibility of the hybrid approach it was first important to obtain a finite element solver which could be given a FE model containing data about forces, materials and the mesh structure and then execute the model programmatically so as to obtain stress results. \\ 

\noindent
A multitude of commercial FE tools exist with there being a wide variety in both the complexity and cost associated with each tool. 
Finite element software is typically very expensive due to its high development cost and small customer base. Tools used within industry such as ANSYS typically require a great deal of time in order to become proficient in their usage and can cost in excess of five thousand pounds a year for a single licence \cite{AnsysCost}. It was therefore important to find a tool which was both affordable while also powerful enough to demonstrate a working prototype of the re meshing method.
 
\subsection{LISA}
After reviewing several FE applications used within industry in addition to a variety of less well known ones used within academia and by hobbyists LISA  was selected as the solver application for which to implement  the systems prototypes. \\ 

\noindent
\textbf{Strengths: }LISA is a FE tool which allows the user to run models of up to 1300 element for free; This was beneficial in allowing me to experiment with the software and gauge the feasibility of my projects concept before requiring a full version. Once at a stage in the project where each problem had been solved for small models containing less than 1300 elements an academic licence for the software was purchased for the projects use. \\

\noindent
LISA also provides a GUI which allows visual inspection of the model and its mesh; This is particularly useful for observing the output of the meshing algorithms which can often provide a human with a much better understanding of how the method has performed and whether or not there are obvious bugs. in the implementation of the meshing procedures \\ 

\noindent
\textbf{Weaknesses: } Due to LISA’s simplicity it does not come with an extensive API allowing for easy programmatic use of its inbuilt features, however it is still possible to interface with LISA through less direct means \cite{LISAManual}. LISA models are stored in .liml files which use XML as a meta mark-up format. The model files contain all the information about the model including the materials used as well as loads and constraints and of course the mesh. It is therefore possible to manipulate a .liml file having parsed its contents before writing a new version of the file which LISA can be called to solve. In order to more easily alter the model it made sense to write a wrapper  for the .liml files to abstract the manipulation of their content.


\subsection{Simulation Data Model}
Writing an API for LISA was the first stage of development for my project for which a design had to be considered. The API was crucial in order to program the more complex aspects using basic operations and avoid having to regularly perform direct string manipulation of the input files in order to manipulate the model. \\

\noindent
When the first re-meshing iteration occurs the system needs to read the input .liml file into an equivalent class model which closely resembles the files schema, diagrams for which can be seen in figure 4 below. Each class in this model contains corresponding data and methods used to represent and manipulate the model. These methods are then used by each of the refinement approaches to easily alter the mesh in a controlled manner. Once the mesh  has been adapted however it is required to be assessed by the modules responsible for validating its quality before finally being written back to a .liml file for LISA to solve on the subsequent iteration. Designing the data model so that it closely resembled the LISA schema not only made the higher level programming less confusing but also made serialisation of the data back to .liml format much simpler  thus reducing the number of bugs arising from inconsistencies between different representations of the same data. \\

\begin{figure}[h!!]
\centering
\begin{subfigure}{.5\textwidth}
  \centering
  \includegraphics[width=0.9\linewidth]{../Graphics/DissoFEProto-Model.jpg}
  \caption{Model classes}
  \label{fig:sub1}
\end{subfigure}%
\begin{subfigure}{.5\textwidth}
  \centering
  \includegraphics[width=0.9\linewidth]{../Graphics/DissoFEProto-ElemNode.jpg}
  \caption{Element and Node classes}
  \label{fig:sub2}
\end{subfigure}
\label{fig:test}
\end{figure}

\begin{figure}
\centering
\begin{subfigure}{.5\textwidth}
  \centering
  \includegraphics[width=0.9\linewidth]{../Graphics/DissoFEProto-ModelAnalysis.jpg}
  \caption{Model Analysis classes}
  \label{fig:sub1}
\end{subfigure}%
\begin{subfigure}{.5\textwidth}
  \centering
  \includegraphics[width=0.9\linewidth]{../Graphics/DissoFEProto-MaterialProps.jpg}
  \caption{Material Property classes}
  \label{fig:sub2}
\end{subfigure}
\label{fig:test}
\caption{Class model to represent .liml file structure used by LISA}
\end{figure}

\noindent
One aspect of the data models design which greatly adds to the systems flexibility is the hierarchical design for representing the various Element types. At the root of this structure is the IElement interface, all new Element types must adhere to this in order for the various refinement methods to request refinement of an element using its class. Implementing the interface are a range of abstract classes such as ``SquareBasedElem" and ``TriangleBasedElem" These classes are designed to contain methods that are generally applicable for calculating metrics and re meshing individual elements where the elements fit this abstract category but their concrete implementation specifies their dimensionality and number of nodes, see Figure 6. This is powerful since computing metrics and performing subdivision for a 3d element is simply a reduction using the code for a 2D element but over every face comprising the 3D one. \\ 

\begin{figure}[h!!]                                                   
  \centerline{\includegraphics[width=150mm, scale=1]{../Graphics/ElementHigerarchyDiagram2.png}}
  \caption{Class diagram showing the hierarchy of element classification within the data model, due to time limitations I was not able to implement the respective classes for triangle and line based elements, to see image representations of each element type within this class diagram refer to element type appendix}
  \label{fig:h-refinementImp}
\end{figure}


\subsection{Remeshing methods approach}
\textbf{Element Refinement: } Delegating refinement of individual elements to their respective classes made it much easier to decouple both the stress and heuristic methods allowing each of them to simply have the task of selecting elements which they considered beneficial to refine before calling the createChildElements() method on that element through the IElement interface. This allowed both of these high level refinement approaches to utilise the same low level functionality thus greatly improving code reuse and simplifying the design such that the high level meshing methods are relatively concise. \\

\noindent
Having reviewed both h-refinement \cite{HandPRefinements} and r-refinement \cite{RRefinement} as techniques for performing element subdivision it was concluded h-refinement was prefferable due to its simplicity and widespread use despite typically being more computationally expensive than r-refinement \cite{HandPRefinements} \cite{RRefinement}. 
Upon finishing subdivision and producing new child elements the refinement process also needed to flatten each tree as seen in figure 7 below to produce a new set of elements which could be stored at root level within the data model. This made the elements more difficult to manage although was another requirement imposed by LISA, which lacks an equivalent concept of element hierarchy within its representation of the mesh. As this task involved the deletion of parent elements and therefore needed to be performed by the ``OptimisationManager'' class responsible for combining and executing both refinement methods. \\

\begin{figure}[!h]
  \centerline{\includegraphics[width=150mm, scale=1]{../Graphics/ElemFlattening.png}}
  \caption{Process of flattening a refined element tree into a single list which can be handed back to LISA for processing}
  \label{fig:h-refinementImp}
\end{figure}

\noindent
\textbf{Stress and Heuristic Refinement: } Having evaluated a variety of approaches from the domains of AI it was concluded that the best approach for delivering a system capable of meeting the requirements and demonstrating effectiveness of a hybrid method would be an implementation of the heuristic expert system described by Dolsak. \\ 

\noindent
One key strength of selecting this approach as the alternative method by which to mesh was clear separation of the underlying AI method from what had to be implemented. This not only meant that focus could be given towards the design, implementation and evaluation of the general purpose system but demonstrated that the meshing procedure can for the most part be interchanged depending on the specific type of finite element analysis. \\ 


\subsection{Input Files}
The system requires three basic input files which should be placed within a directory that is given to the program as a parameter, these are files are:

\begin{itemize}
\item A structural model represented as a .liml file which LISA can solve.
\item An initial stress data file generated manually so the system has a starting point.
\item A JSON file containing important edges and associated meta data as identified by an engineer looking at the model.
\end{itemize}

An example of the content and format for each of these input files can be seen in Appendix B



\subsubsection{Combining methods}
Since each refinement method performed a discrete amount of subdivision every time it was called it made sense when developing a hybrid approach to simply enumerate the possible combinations for how much each method could be applied each iteration resulting in a set of two valued tuples up to some value :

\[ \left\{ (a, b) \,\middle|\, \, a,b \in \mathbb{N}\, \, a,b < k \right\} \]


\colorbox{yellow}{not sure if anyone actually cares about multithreading}

\noindent
Each tuple could then be considered one weighting configuration for combining the two methods and could be executed independent of the others to obtain a set of results for that weighting. Consequently it was possible to improve performance when conducting multiple evaluations through parallelism of the different weighting configurations as experiments onto independent threads. When started each thread creates its own directory which it copies the three input files to and runs for its designating weighting configuration. 




\section{Software Implementation}
The following subsections detail the implementation of the final software solution that has been written to meet the objectives posed previously of this dissertation.

\subsection{Languages and platforms}
The final system has been written entirely using the C\# programming language (version 5.0) with Visual Studio 2015 as the development environment on a Windows 10 system. C\# is an application development language built on the .NET framework. Although any number of programming languages could have been used to implement the solution C\# offered a good compromise for developing a system with both structural rigidity through static typing and object orientation in addition to functionality to allow for rapid prototyping. C\# does this well through use of LINQ, a part of the standard library that provides a large number of higher order functions which allow for operations to be performed over any data structure that implements the built in IEnumerable interface. Given that much of the code within the project performs the same operation on collections of nodes and elements stored in lists, arrays and dictionaries which all implement IEnumerable the ability to write much of the project using this capability dramatically reduced the number of errors encountered and increased development speed.


\subsection{Implementation Methodology}
The growing size of the software meant it was important to work systematically to continuously drive the project in the right direction and avoid the introduction of unnecessary complexity. This was achieved through regularly reviewing and refactoring the code which dramatically helped to reduced the amount of bugs introduced. \\

\noindent
For the duration of the project the spiral methodology was adhered to. This enforced multiple deliverable stages that were concluded with a supervisor meeting every one or two weeks. Adopting the spiral methodology also provided flexibility regarding the order in which tasks were able to take place outside of a spiral iteration. This was necessary when conducting a research driven project where direction of work for subsequent development iterations was largely driven by the  findings of the work in the previous ones. \\

\noindent
Tasks were chosen every week for the project, the number of tasks and their complexity was determined using a combination of factors including their complexity, the criticality of the task e.g. Did it need to be completed for other important tasks to be started and the time available to me as the individual undertaking the project (More tasks typically performed on weeks when less work was due for other modules. \\ 


\subsection{Hierarchical Refinement}
\noindent
Elements within traditional FEA can typically be classified as either triangle or square based elements, each of these provide different strengths and weaknesses when used                                                                                                                                                                                                                                                                                                                                                                                                                                                                         to mesh and solve models. Within industry triangular elements are typically preferable since it is always possible to generate an initial triangular mesh from any arbitrary CAD geometry algorithmically. This is done by simply making smaller triangles until all gaps along the edge of the geometry are filled \cite{DelaunyTriangles}. The same cannot always be said  when meshing using square elements. For proof of the solutions concept however it was concluded that square based elements were preferable to triangular ones since the steps required for a basic refinement are much simpler, just take the corners of an element that already exists, add their corresponding x, y and z components before dividing each component by two to achieve the new midpoint. \\ 

\noindent
In addition to refinement it is also significantly easier to define edges which the ILP rules can be applied to when edges naturally form within a structure through a chain of nodes along the edges of square elements.

%The process of re meshing square based elements 

\noindent
Unfortunately Triangular meshes also generally incur a higher computational cost than an equivalent square element mesh due to added complexity of performing the calculations required to remesh in addition to requiring more elements over a given area to achieve the same accuracy. \\

\noindent
From an implementation standpoint writing a square based remeshing algorithm was also substantially easier if given a mesh of exclusively square elements since the primary task to be performed is to repeatedly divide each element into four sub elements, by contrast methods uses to re mesh triangular meshes are typically more complex and have corresponding initial meshes that are harder to generate manually by human operators \cite{HandMeshing}. \\ 



\subsection{Stress Based Refinement}
To focus meshing in areas of high stress each iteration needed to parse the results file from the previous iterations execution of LISA. LISA result files are in csv format by default and contain the displacements and stresses associated with each node within the model once it has been solved. \\

\noindent
Once the data in the output file has been parsed the nodal values for which displacement is known for can be cross referenced against those in the current model by intersecting the lists of node data on node Ids. An evaluation function is then able to determine whether or not any element handed to it meets the criteria for refinement by simply looking at the sum of the stress at its given nodes. If an element is determined to be over the threshold to justify refinement the elements ``createChildElements()'' method is called to subdivide it further. \\


\subsection{Heuristic/Rule Based Refinement}
Each rule is represented as a function within the implementation, this closely resembles the format presented by Dolsak \cite{DolsakPaper91, DolsakPaper94, appOfILPToFEMeshDesign} \cite{ConsultRuleIntelltSystemFE}. The rules resides within the ``RuleManager'' class and each take a number of the defined edges as parameters. When an instance of the RuleManager is created it parses the edges file provided by the user into a list of edges that the rules can then be executed on. Every rule then checks the properties of a particular edge against properties which have been identified through the ILP learning algorithm as being important when the model executes. In cases where the rules accept more than one edge as an argument the system attempts to apply the rule to each pair of different edges in the edge list giving a time complexity of $O(n^2)$ where n is the total number of defined edges.\\

\noindent
If a rule detects a relationship in the model the edge is assigned a criticality rating as defined by the rule, the value is then used by the meshing procedure to determine how many times it should re mesh the elements along that edge. \\ 
 
The properties that can exist between two edges when compared are the following:
\begin{itemize}
\item Edges opposite one another - the edges run alongside one another closely, look at the distance between each of the corresponding nodes and check whether this distance is less than some threshold amount
\item Edges posses the same form - \colorbox{yellow}{Still need to write about this}
%finish this

\item Edges are considered the same - to meet this requirement both edges must be almost the same length, opposite one another and posses the same form.

\end{itemize}

\noindent
Since this system relied on a persistent definition of edges across multiple refinement iterations another challenge was to correctly redefine edges in terms of the newly created nodes so that after meshing had occurred the rules could be re applied to a refined edge to potentially refine it further.

\begin{figure}[!h]
\centering
\begin{subfigure}{.5\textwidth}
  \centering
  \includegraphics[width=0.9\linewidth]{../Graphics/Rule7Implementation.png}
  \caption{Code implementation of rule 7 provided by Dolsak within the RuleManger class}
  \label{fig:sub1}
\end{subfigure}%
\begin{subfigure}{.5\textwidth}
  \centering
  \includegraphics[width=0.7\linewidth]{../Graphics/Rule7Dolsak.png}
  \caption{Rule 7 as stated by dolsak in his papers \cite{appOfILPToFEMeshDesign}}
  \label{fig:sub2}
\end{subfigure}
\label{fig:test}
\end{figure}


\subsection{Mesh Quality Assessment}
Dittmers rules for computing the quality of both individual elements and the entire mesh are built into their own ``MeshQualtyAssessments" and ``ElementQualityMetrics" classes, the latter of which is encapsulated within an element object, like with refinement this allows each element to assess its own quality removing the need for additional utility classes and static methods. \\

\noindent
%not sure about the last sentence here
Since each element is initialised with the nodes that comprise it, it is also possible to derive all the geometric characteristics and thus its quality metrics upon its initialisation. This allows the metrics for each element to also be calculated upon its initialisation and thus removing the risk of null values being returned when other parts of the system request this information.


%Upon evaluation of the project and concluding that effectiveness of the heuristic relied upon overlap of the %heuristically mesh area with areas of high stress


\subsection{Implementation Challenges}
Implementation of the system was not without its share of challenges, some of which required fundamentally re addressing the approach used to tackle the problem. This section outlines the main instances of such cases during the projects development where as a consequence a notable change to the implementation was made often requiring additional research.

\subsubsection{Fast Node Lookup and Update}
A key requirement for the design of the data model generated by the hierarchical re meshing process was the need to perform fast lookup of nodes already present in the mesh. Lookup is important within the meshing methods as a means of checking whether a node that is about to be created already exists within the model, in the event that no such node already exists a new one can be created however if it does then instead of creating a new node the node that already exists needs to be connected to a node in an adjacent element that is currently being refined. If nodes are not linked correctly form correct elements the physics solver is unable to assume the stress moves through one element to another despite both having nodes at the same coordinates, this results in inaccurate output or potentially an error being thrown by LISA. \\ 

\noindent
This issue arose partly as a result of the systems design, as previously mentioned subdivision for every individual element is the responsibility of that element which from a software engineering perspective is very good since it means the low level meshing process for each different type of element could be written within that elements class. This avoids the need for much heavier generalised refinement classes that would have needed to know how to perform the meshing for all elements in the model at once and for each of the different potential element types. A consequence of this was despite every Element being capable of meshing itself perfectly adjacent elements that also requiring refinement needed the ability to reconnect the new nodes along their edges to those that have been created by the adjacent element, this can be seen below in figure 9. \\ 


\begin{figure}[!h]
  \centerline{\includegraphics[width=100mm , scale=1]{../Graphics/nodeLinking.png}}
  \caption{The need for an element to check for existing adjacent nodes when subdividing itself during refinement,\\ \\
  	Orange Nodes - An original node for one or more elements \\
	Red Nodes - new nodes made by Elem A \\
	Purple Nodes - new nodes made by Elem B \\
  }
  \label{fig:h-refinementImp}
\end{figure}


\noindent
The solution to this problem was to store all the nodes in the mesh model within a C\# dictionary structure a reference to which is passed to each element within the model. The dictionary can be indexed using a Tuple of the x, y and z coordinates for the new potential element which will either return a node already at that location or indicate that no such node exists, in which case that element is then responsible for creating the node as its first instance. Dictionaries in C\# represent a generalised instance of a hash table ensuring that lookup and insert are both constant time on average.

\subsubsection{Sorting Element Nodes}
One issue faced when working with LISA was an interface requirement specified requiring nodes for each type of element to be sorted in a specific geometric order. The general rule for node ordering within LISA is to have them form a perimeter around the edge of an element in 3d space without edges crossing one another internal to the element. \\

\begin{figure}[!h]
  \centerline{\includegraphics[width=50mm , scale=1]{../Graphics/BadlyOrderedNodes.png}}
  \caption{Element with 3d skew resulting in edges between diagonals being shortest by a small amount.
  }
  \label{fig:h-refinementImp}
\end{figure}

\noindent
When addressing this problem for simple models constructed from Quad4 elements the most straightforward approach was to simply traverse each of the nodes in the order specified by LISA and with the resulting traversal list being ordered for LISA. The resulting traversal process resembles the following: \\


\begin{algorithm}[H]
 \While{$\exists node \in UnstortedNodes$}{
  Get distance between current node and the next two nodes in UnstortedNodes\;
  
  \eIf{Nodes left ==  1}{
  
  }

  \eIf{distanceToNode1 < distanceToNode2}{
  currentNode Node1\;
  	sortedNodes.Add(Node1)\;
  	UnstortedNodes.Remove(Node1)\;
   }
   {
    currentNode Node2\;
  	sortedNodes.Add(Node2)\;
  	UnstortedNodes.Remove(Node2)\;
  }
 }
 \caption{How to write algorithms}
\end{algorithm}

\noindent
For the most part this approach was both fast and correct for Quad4 elements although in cases where elements were particularly skewed in 3d space it was sometimes possible for the internal diagonals to be shorter than the actual sides as seen in Figure 9 below. This proved to be a significant flaw in the method and brought about the realisation that a reliable approach would not be able to depend simply upon highly variable properties such as node distances. \\ 

\begin{figure}[!h]
\centering
\begin{subfigure}{.5\textwidth}
  \centering
  \includegraphics[width=0.9\linewidth]{../Graphics/SkewedElementIssues.png}
  \caption{Dividing a Quad4 along planes to establish each node as a corner point}
  \label{fig:sub1}
\end{subfigure}%
\begin{subfigure}{.5\textwidth}
  \centering
  \includegraphics[width=0.7\linewidth]{../Graphics/ElementSkewOnBridge.png}
  \caption{Skew in bridge model elements resulting in rejection of the model by LISA}
  \label{fig:sub2}
\end{subfigure}
\caption{Incorrectly sorted elements arising from failure of traversal routine for skewed elements}
\label{fig:test}
\end{figure}



\noindent
Attempting to arrive at a more general solution focus was directed towards sorting the more complex Hex8 element type as this represented a more complete instance of the problem. Analysis of this led to the realisation that the most important task in sorting nodes for an arbitrary type is to simply establish the corner nodes relative to that type. Having established corners correctly sorting then simply required adding them to a list in the order specified by LISA. \\

\noindent
The subsequent method which was used to successfully establish corners for both Quad4 and Hex8 elements was to split nodes for each element using planes running along the x, y and z axis as can be seen in Figure 9 below.

\begin{figure}[!h]
\centering
\begin{subfigure}{.5\textwidth}
  \centering
  \includegraphics[width=0.9\linewidth]{../Graphics/SortingQuad4.png}
  \caption{Dividing a Quad4 along planes to establish each node as a corner point}
  \label{fig:sub1}
\end{subfigure}%
\begin{subfigure}{.5\textwidth}
  \centering
  \includegraphics[width=0.7\linewidth]{../Graphics/SortingHex8.png}
  \caption{Dividing a Hex8 along planes to establish each node as a corner point}
  \label{fig:sub2}
\end{subfigure}
\caption{Splitting Element points using x, y and z planes in order to perform ordering for LISA}
\label{fig:test}
\end{figure}

\noindent
Although this approach resolved the initial problems resulting from simply trying to traverse the nodes it did not offer a strong general case solution to the problem with the code for a Hex8 element needing to be significantly different and more complex than that of a Quad4 and with the potential for the most complex FE element types such as wedge15, hex20, and pyr13 requiring implementations with even greater number of plane divisions and groupings in order to successfully identify every node. \\ 


\noindent
Having already devised two solutions it seemed likely that there would be some body of research surrounding the problem worth investigating. with research leading to a set of possible alternaives known as convex hull algorithms. As the name suggests the goal of a convex hull algorithm is to generate a convex hull, convex hulls have several definitions but the simplest of these as described by \cite{ConvexHulls} is for a set of points in some space a subset S of those points is convex if for any two points P and Q  inside S the line between the two should also be inside S. This is directly applicable in the case of quad4 elements where the LISA sort order is the convex hull of the points, in the case of more complex elements the algorithm can be applied repeatedly to different faced divided though plane splitting before sorting the nodes at the end with knowledge of node ordering within each individual face. \\ 

\begin{figure}[!h]
  \centerline{\includegraphics[width=100mm , scale=1]{../Graphics/ConvexHullGraphic.png}}
  \caption{Illustration of convex hull definition, imace source: \cite{ConvexHulls}
  }
  \label{fig:h-refinementImp}
\end{figure}


\noindent
Having considered several convex hull algorithms including Graham scan \cite{GrahamScan} $O(n\ log\ n)$ and brute force scan $O(n^4)$ \cite{ConvexHulls} \cite{BruteConvex} before deciding to trial the following C\# implementation of the Monotone Chain algorithm, also $O(n\ log\ n)$ \cite{CSharpConvexHull}. \\ 

\noindent
The Monotone Chain algorithm algorithm was developed shortly after Graham scan and builds upon the concepts introduced in the former. Graham's scan works by initially finding the point in the data set with the lowest y coordinate which can be called P. Having found this point the other points in the set are sorted based on the angle created between them and P. Combining both these steps gives a complexity of $O(n\ log\ n)$, with  $O(n)$ to find P and $O(n\ log\ n)$ to perform a general sort of the angles. Moving through each point in the sorted array Graham scan determines whether moving to this point results in making a right or left hand turn based on the two previous points. If a right turn is made then the second do last point has caused a concave shape which has violated the requirement of the convex hull path. In this scenario is repeated the algorithm excludes the point from the convex set and resumes with the previous two points being those on the path before the rejected point. A stack structure is therefore typically used to keep track of the point ordering as is the case with the Monotone Chain implementation within the system \cite{ChainHull}. \\ 

\noindent
The Monotone Chain algorithm performs essentially the same procedure however instead of sorting using simply y values Monoton Chain sorts using both x and y values. This allows the algorithm to sort the points in two separate groups which form the top and the bottom of the hull and a reduction in the complexity of the sort comparison function. \\ \\ 

%Monotone chain
\begin{algorithm}[H]

%\begin{algorithmic}[1]
%\Procedure{Monotone Chain algorithm}{}


	Sort the points of P by x-coordinate (in case of a tie, sort by y-coordinate)\;
	%\State $\textit{U} \gets \text{Empty List}\textit{string}$
	%\State $\textit{L} \gets \text{Empty List}\textit{string}$
	Make two empty lists I and L
	Lists hold vertices of upper and lower hull\;
	
	\While{i = 1; i < n; i++}{
		\While{L contains at least two points and the sequence of last two points of L and the point P[i] does not make a counter clockwise turn}{
		Remove the last Point from L\;
		}
	}
	
	\While{i = n; i > 1; i- -}{
		\While{L contains at least two points and the sequence of last two points of L and the point P[i] does not make a counter clockwise turn}{
		
		Remove the last Point from U\;
		
		}
	}
	
Remove the last point of each list (it's the same as the first point of the other list).\;
Concatenate L and U to obtain the convex hull of P.\;
Points in the result will be listed in counter-clockwise order.\;

\caption{Monotone Chain algorithm for generating convex hull, pseudocode description credit: \cite{MonotoneChain}}\label{MonotoneChain}
\end{algorithm}

%\cite{MonotoneChain}

%algo 2


\noindent
\newline
%This method has $O(n\ log\ n)$ time complexity however due to the size of n being 4 in all cases the complexity of sorting an individual element is constant, with the overall complexity of sorting all elements in the model being $O(n)$ where n is the number of elements. \\ 

%A key drawback of both Graham scan and Monotone Chain is their limitation to 2D space. Despite the existance of algorithms for generating convex hulls in n dimensions such as \cite{} and

\noindent
The additional complexity of implementing a 3D convex hull algorithm meant it was much easier to experiment the with the approach as a potential solution to the problem using a 2D implementation by simply reducing the problem to a 2D equivalent. This was for quad4 elements by calculating the maximum delta between the max and min value on each axis and eliminating the axis with the smallest delta. These new 2D points could be given to the algorithm which when used in conjunction with the approach already taken was able to solve all instances of node ordering within the models. The only instances in which this approach failed were where highly elements would lie on a perfectly diagonal plane resulting in two axis of elimination using this method. This problem was avoid however by using the basic traversal to sort these elements. \\ 



%Think about putting this under evaluation or removing it instead
\subsubsection{Attempts to Automatically Define Edges Within Models}
As discussed under evaluation another central faced was identifying where the system behaved poorly as a result of weakness in the underlying methods or poor design and implementation  as opposed to poor edge definitions as input for the heuristic method. One way to avoid this issue of evaluation would be to remove all user intervention from the process besides the configuration of the initial model.

\noindent
The obvious way by which to do this was automatic identification of interesting edges which would normally be the responsibility of the trained engineer. Such an approach initially seems promising since the quality of edges as with the mesh can be judged using a small number of key properties. For example it is known that edge importance directly correlates with the size of the edge and how much force is applied near it, since this information exists within the data model it should be possible to identify edges from it and generate them automatically for Dolsaks rules to process.  \\

\noindent
In practice there were multiple complications surrounding this, several of these arise from ambiguity in Dolsaks paper regarding what constitutes a an edge which is for example "long" or "important". As a result edges are only able to be defined to the extent that a user has confidence in their understanding of these concepts. \\ 

Assuming it is possible to generate a system capable of producing edges of interest for a model the problem of evaluating system correctness persists. Bad results then indicate that one of the two systems is broken but provide no clear indication as to which one is.


Having considered a method using these properties of the mesh several days were then spent attempting to implement it with poor success achieved.


\section{Evaluation of Project}
Evaluation of the project consisted of multiple stages ranging from verification of the functional requirements through simple black box testing to evaluation of refined mesh quality using methods provided by Dittmer \cite{DittmerMeshQualityMet}. 

\subsection{Validation Against Functional Requirements}
In order to validate the system against many of the functional requirements the system only needs to be run on several basic models with different input configurations. 

% this
The output clearly demonstrates the systems ability to evaluate the quality of meshes using a range of metrics and refinement occurring as a result of both the stresses induced by the user and on their categorisation of edges. 


\subsection{Validation Against Non Functional Requirements}
Validation of non functional requirements was made simple due to the limited number of them, this was partly a consequence of the system not being designed for a specific user base resulting in expectations regarding the systems design to improve usability and guide interaction. It was also not possible to define the general accuracy and performance of the system during the requirement elicitation phase since this could only be determined once the required research and trial on the finished system gave indication to both of these. \\

\noindent
In the case of quality for the systems design and documentation evidence is present to indicate that this adheres to the requirements specified with the project submission containing detailed documentation in the form of a Deoxygen guide and sophisticated use of object oriented and functional software design as seen within the codebase. General applicability of functionality has also been demonstrated through evaluation using a variety of both models and conditions when performing simulations.

\subsection{Unit Testing}
Holistic evaluation provided evidence of the overall systems effectiveness however without verification of individual components it would not have been possible to assert the accuracy of the results produced. Unit testing was also conducted from within Visual Studio using the NUnit framework and structured as a separate VS project. This Guaranteed that the system was not able to interact with the tests and that testing was conducted through the class and function interfaces provided by the implemented solution. Tests were also grouped into classes with each test class corresponding approximately to one class within the system. Each test class then contains a number of test functions each of which performing the asserts necessary to deem its associated function as correct. This layout provided clear traceability from each item of function to its associated test making assessment of the test coverage much easier. \\

It was important when unit testing not only to write tests for the different aspects of system functionality.

The design and execution of the tests did manage to reveal a bug within the system . 

Much of the software

 this allowed test names to replicate their partner functions making the 

\subsection{Software Quality and Management}
The quality of the design and implementation of the system reflects my experience not only as a computer science undergraduate but as a developer with one year industrial experience, although not directly effecting the execution of the program properties such as appropriate variable naming, loose coupling of classes, use of abstractions and descriptive error messages make the software easier to read and debug for any potential future developers. \\

\noindent
Visual Studio also enabled calculation of various software quality metrics for the code base automatically. This made selecting parts of the codebase for refactoring much easier when time was allocated for this. Upon completion of the project the average maintainability index across all modules was 75 with the lowest score for any high level module being 60 and the highest 92. According to the Microsoft Developer Network (MSDN) website code with an index of between 0 and 9 indicates low maintainability, 10 to 19 indicates moderately maintainable and 20 to 100 high maintainabilty \cite{VisualStudioMaintainIndex}. \\

\noindent
In order to ensure progress was responsibly backed up and new features easily managed a private Github repository was set up and all progress made to the project pushed every couple of days. This proved invaluable on at least one occasion where a bug was accidentally introduced and despite efforts could not be removed manually. 

\subsection{Documentation}
The process of continuously writing descriptive documentation was important to the success of the project and was treated as an integral part to meeting the goals of the project development methodology which aimed to reduce the systems complexity and improve readability. Through the writing Doc comments corresponding to every function within the codebase it was possible to generate documentation files automatically through use of the tool Doxygen. This allows anyone with the solution to view descriptions of each of its functions either in the codebase or alternatively through the manual produced automatically by Doxygen.



\subsection{Increase in performance through parallel execution}
The following 

The average speed up (Time of serial execution/ Time of parallel execution) was 

The Efficiency of the parallel execution (speedup / number of processors) was
 

\subsection{Evaluation Of System For Model Simulations}

%To demonstrate the systems applicability to a variety of real engineering problems demanded the creation of several models resembling basic equivalents of real structures that are often analysed by FE methods.

Several models have been created resembling basic equivalents of that used for real engineering applications.

do this reliably and validate the heuristic component against the results obtained in academic papers two of the three models were based on those previously used by Dolsak (Paper mill and Cylinder structures). In addition to these two a suspension bridge model was developed to demonstrate the systems capacity to work beyond Dolsaks basic validation examples. \\ 

\noindent
These initial models were constructed manually using LISA's graphical user interface using measurements by Dolsak and in the case of the suspension bridge from documents available on the web \cite{DolsakPaper91} \cite{SuspensionBridgeMeasurements}.

\subsubsection{Suspension Bridge structure}

The main model used for testing the methods was a suspension bridge structure. The bridge seemed like a good model to perform initial experimentation on since its structure is 


Constructed manually out of Quad4 LISA elements it has four key constraint points at the base of each supporting column. The bridge itself initially consists of 196 elements and 212 nodes before remeshing, which is considered coarse for such an FE structure. \\


\noindent
When testing the system with the suspension bridge model the first step was to test each method independently of the other in order to evaluate both approaches before attempting to combine the two. Testing each method individually with multiple configurations provided a better understanding of the variation within each allowing more accurate high level reasoning about the results of a hybrid. \\

\noindent
Variation of inputs was particularly important when testing Heuristic method as experimentation revealed a high degree of variation in the results both in terms of runtime and the methods ability to reveal high stress or displacement areas, this can be seen looking at the graph in Figure 14 below and at the meshing results for the structure in appendix F. Four main tests were run for the heuristic approach with each test having a different edge specification file of some level of a designated level of quality. With the very good and good edge specifications I made sure to run the model first using stress refinement and then observe areas of high displacement, this allowed specification of edges around areas I knew in advance would benefit from being specified. The idea being a more experienced FE engineer would be able to do this without having to run the stress refinement first. In addition to the two good quality edge specification sets an OK one was also produced where some edges were specified as existing over the highest stress area while others focused mainly on the lower stress areas, and finally a set of poor edges none of which overlapped the high stress area. \\


\begin{figure}[!h]
\centering
\begin{subfigure}{.5\textwidth}
  \centering
  \includegraphics[width=0.9\linewidth]{../Graphics/Graphs/BridgeCrossLoadingExecutionTimesPerIterations.png}
  \caption{Execution time in seconds for each method to reach 40 refinement iterations}
  \label{fig:sub1}
\end{subfigure}%
\begin{subfigure}{.5\textwidth}
  \centering
  \includegraphics[width=0.9\linewidth]{../Graphics/Graphs/BridgeCrossLoadingAverageDisplacementRevealed.png}
  \caption{Average amount of displacement and thus stress revealed by each approach as number of iterations increases}
  \label{fig:sub2}
\end{subfigure}
\label{fig:test}
  \caption{Execution time increase compared to the amount of information revealed for the different approaches}
 \end{figure}

\noindent
For stress refinement the varied parameter was the threshold used to determine whether elements were considered under high stress and should be further refined. The main consequence of varying this parameter was change in both the node count and the programs runtime as can be seen in Figure 14a. Despite time complexity for subdivision being $O(n^2)$ for all methods using a low threshold results in large values of n and consequently a rapid increase in both runtime and element count. \\

\noindent
In order to perform a reasonable comparison between both the stress and heuristic refinement methods it was important to specify the threshold such that approximately equal amounts were performed by each approach per unit of weighting for each iteration (see combining methods under System Design), thus allowing each method to be evaluated on its merit to select elements appropriately and a hybrid method through its designated weightings. To do this the increase in element count was tracked for the different heuristic methods, as can be seen in Figure 10. The average increase in elements per iteration could then be calculated as 6\% of the total number of elements within the model. This meant stress refinement could be reasonably compared to heuristic refinement if for each use of the refinement it only re meshed those elements considered to be above the 94th percentile in terms of stress. \\ 


\begin{figure}[!h]
  \centerline{\includegraphics[width=110mm, scale=1]{../Graphics/Graphs/BridgeCrossLoadingElementCount.png}}
  \caption{Increase in element count for the various refinement methods over forty iterations}
  \label{fig:sub1}
\end{figure}

\noindent
%Having obtained data from multiple runs of the individual methods 
Executing the bridge model for a variety of hybrid weightings revealed the following general trends in the amount of stress discovered per iteration:

\begin{figure}[!h]
  \centerline{\includegraphics[width=110mm, scale=1]{../Graphics/Graphs/BridgeCrossLoadingAverageStressRevealed.png}}
  \caption{Increase in element count for the various refinement methods over forty iterations}
  \label{fig:sub1}
\end{figure}



\begin{figure}[h!]
\centering
\begin{subfigure}{.5\textwidth}
  \centering
  \includegraphics[width=0.9\linewidth]{../Graphics/BridgeCrossLoadingStress/bridgeStressBasic.png}
  \caption{Stress Revealed through the initial highly coarse bridge mesh without running any iterations for either method}
  \label{fig:sub1}
\end{subfigure}%
\begin{subfigure}{.5\textwidth}
  \centering
  \includegraphics[width=0.9\linewidth]{../Graphics/BridgeCrossLoadingStress/Hybrid-best-3-2.png}
  \caption{Stress revealed within model after just 4 iterations with a heuristic stress method weighting of 3-2, edge heuristics also considered good in this case}
  \label{fig:sub2}
\end{subfigure}
\label{fig:test}
  \caption{Execution time increase compared to the amount of information revealed for the different approaches}
 \end{figure}


Looking at the metric and the resulting meshes it is clear that the system very quickly identifies areas under very high stress with both of the two approaches, with meshing being highly focused on those areas and consequently a rapid increase in the average stress covered by the elements in the model after only two iterations. 

%say something about how actually you want a bit of distribution

\noindent
Comparing the amount of improvement performed by both the stress and heuristic refinement processes with varying weightings that in general heuristics performed better than stress refinement although interestingly more weighting did not necessarily correlate to better results as can be be seen with heuristic weighting of two doing a better job than weighting four in the final iteration. \\


\begin{figure}[!h]
  \centerline{\includegraphics[width=110mm, scale=1]{../Graphics/Graphs/AverageStressDetected Per Iteration By Different Method Weightings.png}}
  \caption{Increase in element count for the various refinement methods over forty iterations}
  \label{fig:sub1}
\end{figure}



\begin{figure}[!h]
  \centerline{\includegraphics[width=110mm, scale=1]{../Graphics/BridgeCrossLoadingStress/BridgeCrossLoadingStress6-3-2.png}}
  \caption{Intense meshing occurring around stress points at joint intersections}
  \label{fig:sub1}
\end{figure}


\noindent
Effective comparison of the two methods indicates that the system works as a tool by which comparisons of different refinement methods can effectively be made. 

%


\subsubsection{Evaluation Issues}
A significant issue faced in attempting to demonstrate the effectiveness of the system was to provide an indication of how well the system worked without taking into account the ability of the user who may be providing the edge rules for a particular model. Not taking this into account would result in an inaccurate representation of its ability.

\subsection{Strengths and Weaknesses}
The resulting system successfully satisfied both the functional and non functional requirements in addition to providing insights into the possibilities of a hybrid technique for effective finite element meshing, something that was optimistic at the start of the project but highly desirable. The project was well managed with all of the objectives being delivered as per the initial time plan. Quality was also maintained throughout the project by the application of good software engineering practice. \\

\noindent
One of the great strengths of the finished system was its modular architecture which allows for a great amount of potential extendibility in the future. Although little focus has so far been given to the systems usability it could be developed and distributed as a public tool for experimentation with hybrid meshing with limited additional effort. \\

\noindent
Another strength of the system is its ability to accept any heuristic definition in terms of edges within a mesh structure. Theoretically this means the final system is also capable of using the same types of edge specifications for any type of FE analysis such as fluid flow or heat transfer given a corresponding rule set by which to mesh with. \\


\noindent
A downside of the current design is the need for the user to manually specify the edges by the user directly into the JSON input file which is both time consuming and prone to error despite the relatively small size of the models analysed in this dissertation. Comparing the size of these with those used in industry it is clear that this process is simply not practical for engineers conducting FE analysis. To change this better tools are required that will allow engineers to automatically generate edge specifications quickly, most likely through some GUI or a bespoke high level language capable of combining knowledge about the mesh structure and different types of edges to generate specific rules. Again this is beyond the scope of the project and would likely be a dissertation in its own right. \\


\noindent
Although the system had a strong subsystem and class level architecture many of its weaknesses could be attributed to needing to prioritise the ability to perform rapid prototyping over efficient implementation of the various algorithms and methods described in this dissertation. Much of this a consequence of overusing the functional programming capabilities within the C\# LINQ library. Widespread adoption of functional programming practices was stated as a desirable aspect of the final system implementation within the non-functional requirements. This has largely been adhered to with  higher order and lambda functions widespread throughout the codebase. In the later stages of the project it became apparent however that in many cases reliance on these features resulted in reduced readability and performance for many of underlying algorithms described in this dissertation. \\


% Overall the project met all its initial requirements laid out in both the objectives and its requirements.
% Sould have done test driven development to ease testing at the end


\subsection{Evaluation summary}


\section{Further Work}
This section details some areas which given additional time to work on the project would at the very least be investigated, if not implemented. Each of these areas would hopefully provide some benefit in assisting to demonstrate the possibilities of hybrid methods.

\subsection{Gathering Feedback From Experienced Engineers}
Approaching the end of the project it became clear that in order to better identify the systems strengths and weaknesses would require additional user testing by engineers who have experience conducting this type of analysis. Despite a lack of available time obtaining feedback from engineers with applied industrial experience along with that of academics this would have allowed for a more conclusive analysis of the systems and its ability to work across a greater number of general case scenarios. 
User feedback was largely not obtained within the duration of the project as a result of time constraints and the complexity inherent in simply implementing and validating the project for a selection of basic models. As such even if time had been available the ethical clearance required to collect user feedback at the start would need to have been acquired.


\subsection{Improving Usability Through A Web Interface}
Although it would have been possible to visit various engineers in order to conduct feedback the process would have been both time consuming on my part and inconvenient for the participant as a time at which to meet must be scheduled, also a laptop containing the working software would need to be brought to them on which they must design or transfer their model to before running it multiple times to obtain results. This scenario is at best inconvenient for the participants and pressures them into arriving at a conclusion within a relatively short period of time after starting to experiment with it. \\ 

\noindent
An alternative approach would be to develop a web interface so as to allow users to interact with the system in a more efficient manner. This approach would allow engineers to submit feedback digitally which could then be aggregated from a much greater range of users sources separated by significant geographical distance. \\\ 

%Instead by facilitating interaction with the system by means of a web interface the engineers would be able spend as little or as much time as they like experimenting with 

\noindent
To use the web interface an engineer would simply need to submit a model they have already created  along with a JSON file containing edges they have designated as important for their model. LISA supports imports from multiple CAD formats including the Standard for the Exchange of Product model data (STEP) and Initial Graphics Exchange Specification (IGES)) \cite{LISAManual}. Upon receiving the request the web server would run the system using their input data and having finished allow them to download the re meshed model along with the calculated stress data for analysis.


\subsection{Added Sophistication of Hybrid Generation}
It has been shown the system can be used to effectively execute and evaluate discrete combinations of different methods it is clear this is an incredibly simple approach to demonstrating the working concept in reality the optimum meshing strategy is likely to be some fuzzy function of several meshing approaches with gradient weighting. As such this would be an exciting direction in which to take the project in future and would greatly increase the experimentation flexibility of the overall system.


\newpage
\section{Project Conclusion and Personal Reflections}
%While conducting research, development and implementation for the project I have worked methodically to best understand each problem as and when they occur and consider each of the possible solutions before committing to one. I feel this approached has saved me much time and has forced me to reach a better view about what exactly each subsystem should do and how to implement it. In hindsight if I were to repeat the first part of the project again I would have organised my time differently to spend less of it focussed on the details of coursework assignments for other modules into making further progress on the implementation of the rule system.

%\noindent
%Overall I felt the project was a success. The final software solution was evidently capable of facilitating execution of the specified functions and therefore allowed comparison to be made of both heuristic and stress based mesh refinement techniques. Not only did the final system allow this comparison for the two specific approaches selected, it provided a flexible framework allowing experimentation for hybrid meshing using any two meshing approaches. \\ 

\noindent
Having used the system to successfully evaluate a range models and compare two individual methods for finite element meshing it has been shown that the project has been delivered to meet each of the three main objectives outlined under ``Description Of The Work''. The delivered system has been demonstrated capable of being able to effectively evaluate meshing approaches using both a traditional refinement approach and one derived from the domain of AI with effective comparisons between each. Simulation results from the suspension bridge above and the paper mill disk and cylinder (appendices I and J) have shown that there are significant potential benefits of using an alternative method such as an expert system in conjunction with traditional stress based refinement and that this can be applied without degradation of quality to the original mesh geometry. Although unlikely that an alternative refinement process will supersede stress based refinement in the near future the high computational cost for large models and the demonstrated potential of alternatives supports the case for conducting further research and development in this area. \\


\noindent
From my own perspective I wanted to use this project as an opportunity to improve my understanding of a technology that I previously had limited knowledge of through its use on my industrial placement year. My prior experience with FE analysis was very much confined to that of a typical engineer making use of the method through a licensed desktop application with many of the technicalities that are of most interest to a computer scientist hidden. I therefore found the project highly enjoyable as an opportunity to learn more about the underlying processes through both research and practical experimentation. As a means of facilitating my personal learning as an individual I therefore also consider the project a success. \\ 

\noindent
Despite working on larger software projects during my year within industry this was certainly the most complex project I have undertaken as an individual. As the lead software developer on my own project I encountered many challenges which as a junior developer within industry were not my responsibility but which I observed team leaders and senior developers encountering regularly. Such tasks were those requiring high level analysis of the design and purpose of the system in order to continuously steer the project in the right direction. In  many such cases the direction the project needed to take was not obvious making it hard to focus purely on implementation. Discussion and management of these decisions with my supervisor Jason Atkin ensured that the project was never stalled for too long and all tasks were successfully delivered within the specified time scales. As a result of these challenges I feel the project has provided me with a much better appreciation of the difficulties associated with delivering a software project in its entirety. \\ 

\noindent
Throughout the majority of the project organisation of time and planning of activities was done well. Work on the project began early with the goal of easing pressure in the later stages and work continued despite deadlines for coursework associated with other modules. A crucial mistake made was to reduce effort two months before the deadline having completed the software implementation and written much of the initial sections of the dissertation despite not completing evaluation of the results. \\

\noindent
The research and evaluation phases were probably the most challenging for me personally, upon finishing I came to realise this was mostly due to a combination of my lack of prior experience with regards to academic research and formal education in mechanical engineering. Both of these factors meant I had to work a lot harder both to understand the initial problems associated with the methods and subsequently to perform reasonable evaluations of both my own results and those described within academic literature. One such example in this was the exponential increase in stress at particular points which took me by surprise having not stressed models to the point of breaking before. Overall had I chosen a more traditional computer science topic I believe both the research and evaluation stages would have been much easier and taken considerably less time. \\ 

\noindent
As the project progressed the increase in scope also presented problems for me as the sole researcher and developer of the system. With a considerable body of research in the wider academic community about each of the specific problems the system needed to solve there was only time for me to survey the most popular papers for each subtopic. This in conjunction with much of the literature being highly specialist and requiring a postdoctoral level of understanding on finite element meshing meant that in the end it was only possible for me to write basic implementations for each of the subsystems given the time available to me. \\ 
  
\noindent
I believe that having completed the research too much time was then spent concerned with the specifics of the implementation, much of which was associated with integrating the functionality of LISA into my system. Although LISAs simplicity was its great strength and helped in simplifying many of the initial design and testing aspects of the project its lack of an extensive API resulted in a large amount of the projects time being focused towards system integration issues which were not apparent during the design and research stages. Although these problems such as element sorting and data modelling proved interesting challenges solving them was considerably more time consuming than was initially predicted and thus reduced the amount of time that could be directed towards the other more theoretical components. Given the chance to repeat the project and having learnt a lot about of finite element systems I feel I would better placed to both use and evaluate a greater range of potential choices. Its likely I would have therefore changed the finite element application and instead tried to use time I may have saved to improve the system for combining methods to add sophistication.

In the end I was glad that I selected C\# as the implementation language and would probably do so again with the exception of perhaps Java so as to have better cross platform compatibility. Initially I was also considering  Python although upon reflection I feel this would have been a mistake with implementation of the more object oriented aspects such as the element interface and subclass structure being made much more difficult by the language.



%what keep what change


\bibliographystyle{abbrv}
\bibliography{bibItems}


\appendix

\section{Element Types within LISA}
Here are shown the the visual specifications LISA provides for the ordering and layout of nodes for defining each type of element supported. Each of these types can be classified using the 




\begin{figure}[!h]
\centering
\begin{subfigure}{.5\textwidth}
  \centering
  \includegraphics[width=0.3\linewidth]{../Graphics/LISA-quad4.png}
  \caption{quad4 element}
  \label{fig:sub1}
\end{subfigure}%
\begin{subfigure}{.5\textwidth}
  \centering
  \includegraphics[width=0.3\linewidth]{../Graphics/LISA-hex8.png}
  \caption{hex8 element}
  \label{fig:sub2}
\end{subfigure}
\label{fig:test}
\caption{Square based elements}
\end{figure}


\begin{figure}[!h]
\centering
\begin{subfigure}{.5\textwidth}
  \centering
  \includegraphics[width=0.3\linewidth]{../Graphics/LISA-tri3.png}
  \caption{tri3 element}
  \label{fig:sub1}
\end{subfigure}%
\begin{subfigure}{.5\textwidth}
  \centering
  \includegraphics[width=0.3\linewidth]{../Graphics/LISA-tet4.png}
  \caption{tet4 element}
  \label{fig:sub2}
\end{subfigure}
\label{fig:test}
\caption{Triangular based elements}
\end{figure}


\begin{figure}[ht]
\centering
\begin{subfigure}{.5\textwidth}
  \centering
  \includegraphics[width=0.3\linewidth]{../Graphics/LISA-line2.png}
  \caption{line2 element}
  \label{fig:sub1}
\end{subfigure}%
\begin{subfigure}{.5\textwidth}
  \centering
  \includegraphics[width=0.3\linewidth]{../Graphics/LISA-line3.png}
  \caption{line3 element}
  \label{fig:sub2}
\end{subfigure}
\label{fig:test}
\caption{Line based elements}
\end{figure} 

%\section{Calculating Centripetal Force For Paper Mill}
%Assuming a constant speed of the paper mill disk at  the following standard calculation was done to compute a forces that could be specified for different elements in order to simulate the effects on the model.

%F = m $\omega^2$ r\\ 

%where: \\ 
%m - mass of object \\ 
%r - radius from centre \\ 
%$\omega$ - angular velocity (radians per second)

%Using the following values for each variable for the plates forming the outside of the paper mill disk the force could be calculated as:

%mass- paper mill is made of steel with each plate having a volume of approximately $24cm^3$ which gives a mass of
%188 grams


\section{Unit Testing}
%Unit tests to validate the correctness of key functionality

\begin{figure}[H]
  \centerline{\includegraphics[width=75mm, scale=0.5]{../Graphics/unitTests.png}}
  \caption{Visual Studio window showing the small suite of twenty tests for validating the core functionalyu of the system.}
\end{figure}

%{Centripetal}

\section{Edge Definition Categories}

\textbf{Edge Type}
\begin{itemize}
  \item important long
  \item important
  \item important short
  \item not important
  \item circuit
  \item half circuit
  \item quarter circuit
  \item short for a hole
  \item long for a hole
  \item circuit hole
  \item half circuit hole
  \item quarter circuit hole
\end{itemize}

\textbf{Boundary Type}
\begin{itemize}
  \item free
  \item fixed on one side
  \item fixed on two sides
  \item fixed completely
\end{itemize}

\textbf{Load Type}
\begin{itemize}
  \item no loading
  \item one side loaded
  \item two sides loaded
  \item continuous loading
\end{itemize}

\section{Input and Output Files}
Below can be seen the format of the input files for the system, a LISA .liml and a .json edge definition file\\

\begin{figure}[H]
\centering
\begin{subfigure}{.5\textwidth}
  \centering
  \includegraphics[width=0.6\linewidth]{../Graphics/limlFileLayout.png}
  \caption{Cut down .liml file to show general content which largely defined the schema for the systems data model}
  \label{fig:sub1}
\end{subfigure}%
\begin{subfigure}{.5\textwidth}
  \centering
  \includegraphics[width=0.8\linewidth]{../Graphics/jsonEdgeFileLayout.png}
  \caption{A json file containing the edges of interest specified by an engineer, this is parsed and the rules are applied to determine the models meshing based on the input}
  \label{fig:sub2}
\end{subfigure}
\label{fig:test}
\end{figure}


\section{Project Layout in Solution Explorer}
Below show the Visual Studio Solution Explorer which provides a general idea of the layout of the project with namespace hierarchies from within an IDE.

\begin{figure}[H]
  \centerline{\includegraphics[width=60mm, height=180mm, scale=0.25]{../Graphics/VSolutionExplorer.png}}
  \caption{The metrics calculated by visual studio for all high level modules in the system}
\end{figure}




\section{Doxygen Documentation}
\begin{figure}[H]
  \centerline{\includegraphics[width=165mm,  scale=0.5]{../Graphics/Doxygen/Quad4Element.png}}
  \caption{The manual page for the Quad4 element class with the class hierarchy and specific public methods}
\end{figure}

\begin{figure}[H]
  \centerline{\includegraphics[width=165mm,  scale=0.5]{../Graphics/Doxygen/RefinementManager.png}}
  \caption{Code for the RefinementManager class viewed within the Doxygen UI}
\end{figure}



\section{Software Quality Metrics}
\begin{figure}[H]
  \centerline{\includegraphics[width=165mm, scale=0.5]{../Graphics/softwareQualityMetrics.png}}
  \caption{The metrics calculated by visual studio for all high level modules in the system}
\end{figure}

\begin{figure}[H]
  \centerline{\includegraphics[width=165mm, scale=0.5]{../Graphics/qualityMetricsExpanded.png}}
  \caption{The metrics calculated by visual studio for the all classes in the final system}
\end{figure}


\section{Mesh Refinements}
This appendix item attempt too show the general mesh that are formed using the heuristic and stress based refinement strategy with examples of where a heuristic has been placed well and poorly and where there is also variation in the threshold used to decide whether elements are meshed with the stress variable. Although in the rest of the models we are looking for stress since this is the primary variable of interest displacement has been selected as the analysis variable for displacement due to it producing a clearer gradient than stress.


\subsection{Heuristic Refinement}



\begin{figure}[H]
\centering
\begin{subfigure}{.5\textwidth}
  \centering
	\includegraphics[width=80mm, scale=0.5]{../Graphics/BridgeCrossLoading/bestEdgeSpecResults.png}
  \caption{Important edges specified effectively to facilitate preemptive meshing of area which undergoes high stress}
  \label{fig:sub1}
\end{subfigure}%
\begin{subfigure}{.5\textwidth}
  \centering
  \centerline{\includegraphics[width=80mm, scale=0.5]{../Graphics/BridgeCrossLoading/bestEdgeSpecResultsCloseUp.png}}
  \caption{Close up view of refinement for high displacement areas }
  \label{fig:sub2}
\end{subfigure}
\label{fig:test}
\end{figure}


\begin{figure}[H]
  \centerline{\includegraphics[width=165mm, scale=0.5]{../Graphics/BridgeCrossLoading/okEdgeSpecResults.png}}
  \caption{Important edges more poorly specified missing high displacement region on top of furthest suspension bridge tower}
\end{figure}

\subsection{Stress Refinement}
\begin{figure}[H]
  \centerline{\includegraphics[width=165mm, scale=0.5]{../Graphics/BridgeCrossLoading/the90thPercentileRefinement.png}}
  \caption{Iterative stress/ displacement refinement method used to focus meshing on the top 6\% most displaced region of model}
\end{figure}

\begin{figure}[H]
  \centerline{\includegraphics[width=165mm, scale=0.5]{../Graphics/BridgeCrossLoading/aboveAverageRefinement2.png}}
  \caption{Iterative refinement of high displacement but with the remesh threshold specified as the average displacement across the whole model. A consequence of this is the gradient of refinement fidelity that can be seen corresponding to the importance of that part of the structure}
\end{figure}


\begin{figure}[H]
  \centerline{\includegraphics[width=165mm, scale=0.5]{../Graphics/BridgeCrossLoading/aboveAverageRefinement.png}}
  \caption{Closer view of very high meshing intensity}
\end{figure}
%\end{changemargin}



\section{Paper Mill Simulation Results}
For the paper mill simulation angular forces were set up around the outside of the disk so as to simulate the effect of the disk rotating at high speed, with it also being pulled outwards in the axial direction. This generated some interesting patches of stress across the main body of the structure which could easily be specified as edge rules. Looking at figure 34 it is possible to see very high range of stress values for stress observable within the model as a consequence of stress concentrating at particular points.

% edge assignement drawign here

\colorbox{yellow}{Still need to add some more to this bit}

\begin{figure}[H]
\centering
\begin{subfigure}{.5\textwidth}
  \centering
  \includegraphics[width=0.9\linewidth]{../Graphics/PaperMillDolsak.jpeg}
  \caption{Half of Cylinder structure described by dolsak in his papers for training ILP system}
  \label{fig:sub1}
\end{subfigure}%
\begin{subfigure}{.5\textwidth}
  \centering
  \includegraphics[width=0.9\linewidth]{../Graphics/PaperMillWithinLisa.jpeg}
  \caption{Replication of mesh structure specified by Dolsak within his paper \cite{Dolsak91}}
  \label{fig:sub2}
\end{subfigure}
\label{fig:test}
  \caption{Execution time increase compared to the amount of information revealed for the different approaches}
 \end{figure}
 
 \begin{figure}[H]
  \centerline{\includegraphics[width=120mm, scale=0.5]{../Graphics/PaperMillStress/PaperMillFirstUWMesh.png}}
  \caption{The initially stressed paper mill part used to define edge sets for further meshing, stress concentrations can be observed in red with colour coding at the top indicating showing a rapidly exponential increase concentrated at those points}
\end{figure}


\begin{figure}[H]
  \centerline{\includegraphics[width=120mm, scale=0.5]{../Graphics/Graphs/PaperMillExecutionTimes.png}}
  \caption{Time taken per iteration using the different hybrid weightings with varying edge quality specifications}
\end{figure}

 \begin{figure}[H]
  \centerline{\includegraphics[width=120mm, scale=0.5]{../Graphics/Graphs/PaperMillAverageStressRevealed.png}}
  \caption{Improvement in average stress detected across all nodes within the model over multiple iterations, results for each weighting with different edge heuristics also averaged}
\end{figure}


\section{Half Cylinder Simulation Results}
The half cylinder was the third model used to test the system. 




\begin{figure}[H]
\centering
\begin{subfigure}{.5\textwidth}
  \centering
  \includegraphics[width=0.9\linewidth]{../Graphics/HalfCylinder/DolsakCylinderMeshed.jpeg}
  \caption{Half of Cylinder structure described by dolsak in his papers for training ILP system}
  \label{fig:sub1}
\end{subfigure}%
\begin{subfigure}{.5\textwidth}
  \centering
  \includegraphics[width=0.9\linewidth]{../Graphics/HalfCylinder/DolsakCylinderWithinLisa.jpeg}
  \caption{Replication of mesh structure specified by Dolsak within his paper \cite{Dolsak91}}
  \label{fig:sub2}
\end{subfigure}
\label{fig:test}
  \caption{Execution time increase compared to the amount of information revealed for the different approaches}
 \end{figure}
 
 
 \begin{figure}[H]
\centering
\begin{subfigure}{.5\textwidth}
  \centering
  \includegraphics[width=0.9\linewidth]{../Graphics/HalfCylinder/ForcesAndConstraintsOnCylinderPNG.png}
  \caption{Cylinder Constrained by its adjacent half with forces applied up and outwards on its inner rim}
  \label{fig:sub1}
\end{subfigure}%
\begin{subfigure}{.5\textwidth}
  \centering
  \includegraphics[width=0.9\linewidth]{../Graphics/HalfCylinder/InitialStress.png}
  \caption{Half cylinder with an initial stress concentration having performed a simple execution of the model.}
  \label{fig:sub2}
\end{subfigure}
\label{fig:test}
  \caption{Initial configuration for the half cylinder and some stresses revealed on the structure}
 \end{figure}
 
\begin{figure}[H]
  \centerline{\includegraphics[width=120mm, scale=0.5]{../Graphics/HalfCylinder/ImprovementInMaxCornerAngles.png}}
  \caption{Improvement in corner angles for Quad4 elements using the different hybrid methods on the cylinder}
\end{figure}

\begin{figure}[H]
  \centerline{\includegraphics[width=120mm, scale=0.5]{../Graphics/HalfCylinder/AverageStressDetectedUsingAverageOverHeuristicsForDifferentHybrids(HalfCylinder).png}}
  \caption{Stress revealed for each iteration using the different hybrid methods}
\end{figure}

\begin{figure}[H]
  \centerline{\includegraphics[width=120mm, scale=0.5]{../Graphics/HalfCylinder/ExecutionTimes.png}}
  \caption{Time taken per iteration using the different hybrid weightings with varying edge quality specifications}
\end{figure}

 
\section{Gantt Chart for Project Time Management}
\pagestyle{empty}
\begin{landscape}
\vspace*{1cm}
\hspace*{-3cm}
\begin{figure}[H]
\includegraphics[width =700px, height=300px]{../Graphics/TimePlanUpdated2.png} \par
\end{figure}
\end{landscape}



\end{document}


