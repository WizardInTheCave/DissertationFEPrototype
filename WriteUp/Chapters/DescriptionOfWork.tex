
\section{Description of the work}
\subsection{Aims and Objectives}
The aim of this project was to design, build and analyse a system for refining a finite element mesh by combining a method derived from the fields of AI or machine learning with a method relying purely on information already present in the model. The desirable end result will be a hybrid method of meshing which effectively prioritises those areas of importance whilst incurring a reduced computational cost.\\

\noindent
The project can be broken down into thee main areas of research and implementation which have the following high level objectives:\\ 

\begin{enumerate}[label=\Alph*]

\item Research and implement both a traditional refinement procedure using data present within the model and an approach using techniques from AI developed and used by either industry or academia. These algorithms should be able to run independently on a set of example models.

\item Secondly a process needs to be devised to combine the two re meshing methods to varying degrees. This will make it possible to evaluate and compare the effects of a hybrid meshing against each of the individual methods for a range of models. Through this it should be possible to establish whether or not there is potential benefit to using a hybrid approach and if so to what extent.

\item The third objective will be to research and implement justifiable metrics for assessing the quality of a given mesh, this will allow objective comparisons to be made for the resulting meshes.


% This will allow for a much greater range of hybrid variations to be tried without requiring inspection from an expert. 
\end{enumerate}

\noindent
An outline of the initial project plan along with a corresponding Gantt chart can be seen in appendix M. \\

\subsection{System specification}
To demonstrate success in achieving the objectives of the project it is important to have traceability from the requirements through to the solution and lastly verification and validation. Appendix A describes the systems initial \textit{functional requirements} (what the system will do) and \textit{non-functional} requirements (its constraints) based upon evaluation of the research conducted in conjunction with discussions with the project supervisor: Dr Jason Atkin. Functional requirements have primarily been listed under their respective high level subsystems that are responsible for encapsulating their functionality. \\ 

\noindent
Although it has not been developed as part of the project the application responsible for solving the finite element models has been included as part of the systems requirements since it highly influences the overall scope of the project and much of the design associated with other subsystems which were developed for the project. \\ 