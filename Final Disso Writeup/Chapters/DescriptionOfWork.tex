
\section{Description of the work}
\subsection{Aims and Objectives}
The aim of this project was to design, build and analyse a system for refining a mesh by combining a method from the domain of AI or machine learning with a stress based refinement method. The desirable end result will be a hybrid method of meshing which produces reasonably good solution accuracy whilst requiring less user intervention and reduced computational cost.\\

\noindent
The project can be broken into thee main sections of research and implementation which can each be considered a high level objective:\\ 

\begin{enumerate}[label=\Alph*]

\item The first objective is to research and develop both a traditional stress refinement and an AI re meshing algorithm developed by industry or by academia. These algorithms should be able to run independently on a set of example models.

\item Secondly a process will need to be devised for combining the two re meshing methods to varying degrees will be required. This will make it possible to test the effects of a hybrid method across the same set of models. Through this it can be established whether there is notable benefit to combining the different approaches and if so to what degree is it effective.

\item The third objective will be to use justifiable metrics for assessing the quality of a given mesh. This will allow for a much greater range of hybrid variations to be tried without requiring inspection from an expert. 
\end{enumerate}

\subsection{System specification}
To demonstrate success in achieving the objectives of the project it is important to have tractability from the requirements through to the solution and lastly verification and validation. This section describes the current systems \textit{functional requirements} (what the system will do) and \textit{non-functional} requirements (its constraints) based upon evaluation of the research I have conducted in conjunction with discussions with the project supervisor: Dr Jason Atkin. Functional requirements have primarily been listed under their respective high level subsystems that are responsible for encapsulating their functionality. \\ 

\noindent
Although it has not been developed as part of the project the application responsible for solving the finite element models has been included as part of the systems requirements since it highly influences the overall scope of the project and much of the design associated with other subsystems which were developed for the project. \\ 

\subsection{Functional Requirements}

\begin{enumerate}
\item FE integration: System will be able to interface with a third party finite element application

\begin{enumerate}
\item The finite element applications solver must be able to solve a mesh based on its model configuration.
\item The finite element applications solver must be able to execute a model programmatically
\item The finite element applications solver must be able to output stress data at different points on the mesh.
\item The finite element application will provide a graphical representation of the model.
\item It will be possible to manipulate the model that the finite element application uses programmatically.
\item It should be possible to manipulate the model that the finite element application uses from within its graphical user interface.
\end{enumerate}

\item Mesh refinement: System will be able to perform different kinds of finite element mesh refinement

\begin{enumerate}
\item The system will be able to refine a finite element mesh using a stress based refinement method
\item The system will be able to refine a finite element mesh using a non-stress based refinement method

\item A non stress based refinement method will adapt the mesh using background information about mesh design which has been previously trained.

\item The system will be able to combine the two methods to produce a coherent mesh which the FE application is able to successfully solve in order to obtain results for stress and displacement.
\item The system will be able to combine both methods to varying degrees will be performed automatically by the system without direct user intervention.
\item The system will re mesh using both stress and non-stress based refinement using quadrilateral elements.
\item System will adapt weighting associated with each method based upon the metrics computer for the mesh in the systems previous iteration.
\end{enumerate}

\item Quality assessment: System will provide the operator with results about the quality of meshes based on metrics obtained from research.

\begin{enumerate}
\item An assessment will be conducted automatically for every mesh iteration that occurs.
\item System will assess quality based on a variety of metrics to ensure overall robustness of measurement. 
\item The metrics will be computed for both individual elements within the model and for the entire mesh.
\end{enumerate}
\end{enumerate}

\subsection{Non-Functional Requirements}


\textbf{Design:} The system architecture will be developed using the object oriented design principals of SOLID to allow for clear interfaces between the different functional components. Functional programming practices will be adopted through use of the .NET Language Integrated Query or LINQ framework. This will help to simplify the code and improve reliability. Where functions and classes are written their length will be kept to a minimum to reduce complexity and allow for reuse wherever possible. \\ 

\noindent
\textbf{Documentation:} The system will be comprehensively documented at both a code level and at an architecture level. At a code level C\# doc comments will be written to provide a comprehensive summary of each function. This will allow the tool Doxygen \cite{Doxygen} to generate a full set of developer documentation upon completion of the software implementation which will be included as an appendix. Doc comments will also help to encourage small functions. \\

\noindent
\textbf{General applicability:} In order to demonstrate that hybrid methods are a feasible means of approaching meshing problems the resulting software should be able to successfully execute on a range of models with varying geometry. The range of geometries should be representative of typical structural variation.