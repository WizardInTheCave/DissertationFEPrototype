\section{System Design}

\subsection{System Overview}
Determining the overall design of the system was initially hard since it was not clear exactly how many subsystems would be needed to mesh, evaluate and interface with the finite element solver (see section 5.4 on LISA, the selected solver), what was clear was that the system would essentially be performing an optimisation procedure and as such needed to be driven iteratively towards a goal. The variable complexity and uncertainty surrounding the different parts of the project meant ensuring the architecture remained modular with well defined interfaces allowing components to easily be added or modified as the project progressed. \\ 

\noindent
The system can be broken down into several main tasks that form the overall process: \\ 

\noindent
\textbf{Generate Initial Model: } Using an initial input file containing a FEM build a class model which can easily be manipulated by the methods within the code. \\ 

\noindent
\textbf{Write Model To Solver: } Write Model to a solver file that can be solved and call solver to obtain an output file for the model.  \\ 

\noindent
\textbf{Read Solver Output: } Read the stress data computed across the mesh from the output file back into the data model so it can be cross referenced with the current mesh \\ 

\noindent
\textbf{Apply Refinement Processes : } Apply both the stress and alternative refinement process to improve the mesh such that on the subsequent iteration more stress is revealed. \\ 

\noindent
\textbf{Evaluate Refined Mesh : } See how much stress was revealed in the last iteration by applying refinement and whether or not the quality of the elements improved.



\begin{figure}[H]
  \centerline{\includegraphics[width=150mm, scale=1]{../Graphics/OverallProcess.png}}
  \caption{High level process of system}
  \label{fig:h-refinementImp}
\end{figure}



\begin{figure}[H]
  \centerline{\includegraphics[width=150mm, scale=1]{../Graphics/SystemDesignDiagram.jpeg}}
  \caption{High level design of the system with its different modules}
  \label{fig:h-refinementImp}
\end{figure}




\subsection{Modular Architecture}
The modular architecture was crucial for allowing meshing algorithms and quality metrics to be replaced as necessary. At best the quality of the output could could only be predicted for each method before it was integrated into the system and executed in a range of different scenarios. To have tightly coupled these individual components would have rendered the overall system a failure in the event that any one of them failed. Instead the loose coupling of the architecture has enabled the system to be considered as more of a framework for testing the effects of combining different meshing approaches in order to generate a hybrid method.\\

\noindent
Although the system was highly modular It was also still desirable to maintain an architecture hierarchy so that classes could be developed independently but easily integrated. Composition was therefore generally favoured over inheritance as a means of building the architecture. Static classes and methods were also used when needing to write utility functions that were required by multiple high level subsystems and therefore did no fit especially well into any particular one. Examples of these are generic vector algebra operations such as dot product, matrix determinant and calculating surface normals.

\noindent
At the highest level namespaces were used to break down the class groups appropriately, namespaces also naturally structured as folders within the Visual Studio (VS) solution explorer (see appendix D) which made navigating the project and finding components much easier as the system expanded in size.

%\subsection{Mesh improvement Loop}
%As with many optimisation problems the refinement process is driven iteratively through a loop. Within the main loop all %interfacing with LISA, Mesh refinement and analysis is conducted which results in an updated version of the model that can be %handed to the subsequent iteration.
\subsection{Third Party FE Application}
In order to demonstrate the potential feasibility of the hybrid approach it was first important to obtain a finite element solver which could be given a FE model containing data about forces, materials and the mesh structure and then execute the model programmatically so as to obtain stress results. \\ 

\noindent
A multitude of commercial FE tools exist with there being a wide variety in both the complexity and cost associated with each tool. 
Finite element software is typically very expensive due to its high development cost and small customer base. Tools used within industry such as ANSYS typically require a great deal of time in order to become proficient in their usage and can cost in excess of five thousand pounds a year for a single licence \cite{AnsysCost}. It was therefore important to find a tool which was both affordable while also powerful enough to demonstrate a working prototype of the re meshing method.
 
\subsection{LISA}
After reviewing several FE applications used within industry in addition to a variety of less well known ones used within academia and by hobbyists LISA  was selected as the solver application for which to implement  the systems prototypes. \\ 

\noindent
\textbf{Strengths: }LISA is a FE tool which allows the user to run models of up to 1300 element for free; This was beneficial in allowing me to experiment with the software and gauge the feasibility of my projects concept before requiring a full version. Once at a stage in the project where each problem had been solved for small models containing less than 1300 elements an academic licence for the software was purchased for the projects use. \\

\noindent
LISA also provides a GUI which allows visual inspection of the model and its mesh; this is particularly useful for observing the output of the meshing algorithms which can often provide a human with a much better understanding of how the method has performed and whether or not there are obvious bugs. in the implementation of the meshing procedures \\ 

\noindent
\textbf{Weaknesses: } Due to LISA’s simplicity it does not come with an extensive API allowing for easy programmatic use of its inbuilt features, however it is still possible to interface with LISA through less direct means \cite{LISAManual}. LISA models are stored in .liml files which use XML as a meta mark-up format. The model files contain all the information about the model including the materials used as well as loads and constraints and of course the mesh. It is therefore possible to manipulate a .liml file having parsed its contents before writing a new version of the file which LISA can be called to solve. In order to more easily alter the model it made sense to write a wrapper  for the .liml files to abstract the manipulation of their content. \\ 


\subsection{Simulation Data Model}
Writing an API for LISA was the first stage of development for my project for which a design had to be considered. The API was crucial in order to program the more complex aspects using basic operations and avoid having to regularly perform direct string manipulation of the input files in order to manipulate the model. \\

\noindent
When the first re-meshing iteration occurs the system needs to read the input .liml file into an equivalent class model which closely resembles the files schema, diagrams for which can be seen in figure 4 below. Each class in this model contains corresponding data and methods used to represent and manipulate the model. These methods are then used by each of the refinement approaches to easily alter the mesh in a controlled manner. Once the mesh  has been adapted however it is required to be assessed by the modules responsible for validating its quality before finally being written back to a .liml file for LISA to solve on the subsequent iteration. Designing the data model so that it closely resembled the LISA schema not only made the higher level programming less confusing but also made serialisation of the data back to .liml format much simpler  thus reducing the number of bugs arising from inconsistencies between different representations of the same data. \\

\begin{figure}[h!]
\centering
\begin{subfigure}{.5\textwidth}
  \centering
  \includegraphics[width=0.9\linewidth]{../Graphics/DataModelStructure/Model.png}
  \caption{Model classes}
  \label{fig:sub1}
\end{subfigure}%
\begin{subfigure}{.5\textwidth}
  \centering
  \includegraphics[width=0.9\linewidth]{../Graphics/DataModelStructure/Elements.png}
  \caption{Element Classes}
  \label{fig:sub2}
\end{subfigure}
\label{fig:test}
\end{figure}

\begin{figure}[H]
\centering
\begin{subfigure}{.5\textwidth}
  \centering
  \includegraphics[width=0.9\linewidth]{../Graphics/DataModelStructure/ModelAnalysis.png}
  \caption{Model Analysis classes}
  \label{fig:sub1}
\end{subfigure}%
\begin{subfigure}{.5\textwidth}
  \centering
  \includegraphics[width=0.9\linewidth]{../Graphics/DataModelStructure/MaterialProps.png}
  \caption{Material Property classes}
  \label{fig:sub2}
\end{subfigure}
\label{fig:test}
\caption{Class model to represent .liml file structure used by LISA}
\end{figure}






\noindent
One aspect of the data models design which greatly adds to the systems flexibility is the hierarchical design for representing the various Element types. At the root of this structure is the IElement interface, all new Element types must adhere to this in order for the various refinement methods to request refinement of an element using its class. Implementing the interface are a range of abstract classes such as ``SquareBasedElem" and ``TriangleBasedElem" These classes are designed to contain methods that are generally applicable for calculating metrics and re meshing individual elements where the elements fit this abstract category but their concrete implementation specifies their dimensionality and number of nodes, see figure 6 below. This is powerful since computing metrics and performing subdivision for a 3D element is simply a reduction using the code for a 2D element but over every face comprising the 3D one. \\ 


\pagestyle{empty}
\begin{landscape}

\begin{figure}[h!!]                                                   
  \centerline{\includegraphics[width=150mm, scale=1]{../Graphics/ElementHigerarchyDiagram2.png}}
  \caption{Class diagram showing the hierarchy of element classification within the data model, due to time limitations I was not able to implement the respective classes for triangle and line based elements, to see image representations of each element type within this class diagram refer to element type appendix}
  \label{fig:h-refinementImp}
\end{figure}
\end{landscape}


\subsection{Remeshing methods approach}
The strategy adopted for performing refinement using multiple methods was to separate the underlying process of subdividing individual elements (Element Refinement) from the higher level strategies responsible for determining which parts of the mesh structure were most preferable to refine (Stress and Heuristic Refinement). Delegating refinement of individual elements to their respective classes made it much easier to decouple both the stress and heuristic methods. Having selected elements considered beneficial to refine the stress and heuristic methods only need to call the createChildElements() method on that element through the IElement interface. This allowed both of these high level refinement approaches to utilise the same low level functionality for any type of element, see appendix A for element types.  This greatly improved code reuse and simplified the design such that new high level approaches can easily be added without needing to consider how the underlying meshing takes place for models constructed using different elements. \\ 

\noindent
\textbf{Element Refinement: }  Having reviewed both h-refinement \cite{HandPRefinements} and r-refinement \cite{RRefinement} as techniques for performing element subdivision it was concluded h-refinement was preferable due to its simplicity and widespread use despite typically being more computationally expensive than r-refinement \cite{HandPRefinements} \cite{RRefinement}. 
Upon finishing subdivision and producing new child elements the refinement process also needed to flatten each tree as seen in figure 7 below to produce a new set of elements which could be stored at root level within the data model. This made the elements more difficult to manage although was another requirement imposed by LISA, which lacks an equivalent concept of element hierarchy within its representation of the mesh. As this task involved the deletion of parent elements and therefore needed to be performed by the ``RefinementManager'' class responsible for combining and executing both refinement methods. \\

\begin{figure}[!h]
  \centerline{\includegraphics[width=150mm, scale=1]{../Graphics/ElemFlattening.png}}
  \caption{Process of flattening a refined element tree into a single list which can be handed back to LISA for processing}
  \label{fig:h-refinementImp}
\end{figure}

\noindent
\textbf{Stress and Heuristic Refinement: } Having evaluated a variety of approaches from the domains of AI it was concluded that the best approach for delivering a system capable of meeting the requirements and demonstrating effectiveness of a hybrid method would be an implementation of the heuristic expert system described by Dolsak. \\ 

\noindent
One key strength of selecting this approach as the alternative method by which to mesh was clear separation of the underlying AI method from what had to be implemented. This not only meant that focus could be given towards the design, implementation and evaluation of the general purpose system but demonstrated that the meshing procedure can for the most part be interchanged depending on the specific type of finite element analysis. \\ 




\subsection{Input Files}
The system requires three basic input files which should be placed within a directory that is given to the program as a parameter, these files are:

\begin{itemize}
\item A structural model represented as a .liml file which LISA can solve.
\item An initial stress data file generated manually so the system has a starting point.
\item A JSON file containing important edges and associated meta data as identified by an engineer looking at the model.
\end{itemize}

An example of the content and format for each of these input files can be seen in appendix B


\subsection{Combining methods}
Since each refinement method performed a discrete amount of subdivision every time it was called it made sense when developing a hybrid approach based on the two methods to define each potential hybrid as some weighted combination of the two methods. The simplest way to represent this appeared to be a two valued tuple containing the number of times each method should be applied for each iteration. One way to do this automatically would be to iterate through combinations of two integers up to some value k:

\[ \left\{ (HeuristicRefinementIterations, StressRefinementIterations) \,\middle|\, \, a,b \in \mathbb{N}\, \, a,b < k \right\} \]

\noindent
Using this weighting system each application of a method can be thought of as adding a depth of one to element refinement trees affected by the rule as seen in figure 7. Testing the hybrid weightings as different specifications meant it was also possible to improve the systems throughput by conducting evaluations simultaneously on different threads. When started each thread creates its own directory which it copies the three input files to and runs for its designating weighting configuration. \\

\noindent
Another key consideration when comparing the different meshing approaches was to establish what the value of weighting unit and thus allowing comparisons and evaluation for each of the hybrids as a weighting specification. Balance of the weightings was achieved at the start of the evaluation process through observing the increase in element count for the different heuristic methods when run individually as can be seen in Figure 10. With the average increase in element count per iteration being calculated as 6\% of the model total the stress refinement threshold could then be configured so as only to mesh those elements within the 94th percentile within the model in terms of stress. \\

