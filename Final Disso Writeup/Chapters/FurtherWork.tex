
\section{Further Work}
This section details some areas which given additional time to work on the project would at the very least be investigated, if not implemented. Each of these areas would hopefully provide some benefit in assisting to demonstrate the possibilities of hybrid methods.

\subsection{Gathering Feedback From Experienced Engineers}
Approaching the end of the project it became clear that in order to better identify the systems strengths and weaknesses would require additional user testing by engineers who have experience conducting this type of analysis. Despite a lack of available time obtaining feedback from engineers with applied industrial experience along with that of academics this would have allowed for a more conclusive analysis of the systems and its ability to work across a greater number of general case scenarios. 
User feedback was largely not obtained within the duration of the project as a result of time constraints and the complexity inherent in simply implementing and validating the project for a selection of basic models. As such even if time had been available the ethical clearance required to collect user feedback at the start would need to have been acquired.


\subsection{Improving Usability Through A Web Interface}
Although it would have been possible to visit various engineers in order to conduct feedback the process would have been both time consuming on my part and inconvenient for the participant as a time at which to meet must be scheduled, also a laptop containing the working software would need to be brought to them on which they must design or transfer their model to before running it multiple times to obtain results. This scenario is at best inconvenient for the participants and pressures them into arriving at a conclusion within a relatively short period of time after starting to experiment with it. \\ 

\noindent
An alternative approach would be to develop a web interface so as to allow users to interact with the system in a more efficient manner. This approach would allow engineers to submit feedback digitally which could then be aggregated from a much greater range of users sources separated by significant geographical distance. \\\ 

%Instead by facilitating interaction with the system by means of a web interface the engineers would be able spend as little or as much time as they like experimenting with 

\noindent
To use the web interface an engineer would simply need to submit a model they have already created  along with a JSON file containing edges they have designated as important for their model. LISA supports imports from multiple CAD formats including the Standard for the Exchange of Product model data (STEP) and Initial Graphics Exchange Specification (IGES)) \cite{LISAManual}. Upon receiving the request the web server would run the system using their input data and having finished allow them to download the re meshed model along with the calculated stress data for analysis.


\subsection{Added Sophistication of Hybrid Generation}
It has been shown the system can be used to effectively execute and evaluate discrete combinations of different methods it is clear this is an incredibly simple approach to demonstrating the working concept in reality the optimum meshing strategy is likely to be some fuzzy function of several meshing approaches with gradient weighting. As such this would be an exciting direction in which to take the project in future and would greatly increase the experimentation flexibility of the overall system.


\newpage
\section{Project Conclusion and Personal Reflections}
%While conducting research, development and implementation for the project I have worked methodically to best understand each problem as and when they occur and consider each of the possible solutions before committing to one. I feel this approached has saved me much time and has forced me to reach a better view about what exactly each subsystem should do and how to implement it. In hindsight if I were to repeat the first part of the project again I would have organised my time differently to spend less of it focussed on the details of coursework assignments for other modules into making further progress on the implementation of the rule system.

%\noindent
%Overall I felt the project was a success. The final software solution was evidently capable of facilitating execution of the specified functions and therefore allowed comparison to be made of both heuristic and stress based mesh refinement techniques. Not only did the final system allow this comparison for the two specific approaches selected, it provided a flexible framework allowing experimentation for hybrid meshing using any two meshing approaches. \\ 

\noindent
Having used the system to successfully evaluate a range models and compare two individual methods for finite element meshing it has been shown that the project has been delivered to meet each of the three main objectives outlined under ``Description Of The Work''. The delivered system has been demonstrated capable of being able to effectively evaluate meshing approaches using both a traditional refinement approach and one derived from the domain of AI with effective comparisons between each. Simulation results from the suspension bridge above and the paper mill disk and cylinder (appendices I and J) have shown that there are significant potential benefits of using an alternative method such as an expert system in conjunction with traditional stress based refinement and that this can be applied without degradation of quality to the original mesh geometry. Although unlikely that an alternative refinement process will supersede stress based refinement in the near future the high computational cost for large models and the demonstrated potential of alternatives supports the case for conducting further research and development in this area. \\


\noindent
From my own perspective I wanted to use this project as an opportunity to improve my understanding of a technology that I previously had limited knowledge of through its use on my industrial placement year. My prior experience with FE analysis was very much confined to that of a typical engineer making use of the method through a licensed desktop application with many of the technicalities that are of most interest to a computer scientist hidden. I therefore found the project highly enjoyable as an opportunity to learn more about the underlying processes through both research and practical experimentation. As a means of facilitating my personal learning as an individual I therefore also consider the project a success. \\ 

\noindent
Despite working on larger software projects during my year within industry this was certainly the most complex project I have undertaken as an individual. As the lead software developer on my own project I encountered many challenges which as a junior developer within industry were not my responsibility but which I observed team leaders and senior developers encountering regularly. Such tasks were those requiring high level analysis of the design and purpose of the system in order to continuously steer the project in the right direction. In  many such cases the direction the project needed to take was not obvious making it hard to focus purely on implementation. Discussion and management of these decisions with my supervisor Jason Atkin ensured that the project was never stalled for too long and all tasks were successfully delivered within the specified time scales. As a result of these challenges I feel the project has provided me with a much better appreciation of the difficulties associated with delivering a software project in its entirety. \\ 

\noindent
Throughout the majority of the project organisation of time and planning of activities was done well. Work on the project began early with the goal of easing pressure in the later stages and work continued despite deadlines for coursework associated with other modules. A crucial mistake made was to reduce effort two months before the deadline having completed the software implementation and written much of the initial sections of the dissertation despite not completing evaluation of the results. \\

\noindent
The research and evaluation phases were probably the most challenging for me personally, upon finishing I came to realise this was mostly due to a combination of my lack of prior experience with regards to academic research and formal education in mechanical engineering. Both of these factors meant I had to work a lot harder both to understand the initial problems associated with the methods and subsequently to perform reasonable evaluations of both my own results and those described within academic literature. One such example in this was the exponential increase in stress at particular points which took me by surprise having not stressed models to the point of breaking before. Overall had I chosen a more traditional computer science topic I believe both the research and evaluation stages would have been much easier and taken considerably less time. \\ 

\noindent
As the project progressed the increase in scope also presented problems for me as the sole researcher and developer of the system. With a considerable body of research in the wider academic community about each of the specific problems the system needed to solve there was only time for me to survey the most popular papers for each subtopic. This in conjunction with much of the literature being highly specialist and requiring a postdoctoral level of understanding on finite element meshing meant that in the end it was only possible for me to write basic implementations for each of the subsystems given the time available to me. \\ 
  
\noindent
I believe that having completed the research too much time was then spent concerned with the specifics of the implementation, much of which was associated with integrating the functionality of LISA into my system. Although LISAs simplicity was its great strength and helped in simplifying many of the initial design and testing aspects of the project its lack of an extensive API resulted in a large amount of the projects time being focused towards system integration issues which were not apparent during the design and research stages. Although these problems such as element sorting and data modelling proved interesting challenges solving them was considerably more time consuming than was initially predicted and thus reduced the amount of time that could be directed towards the other more theoretical components. Given the chance to repeat the project and having learnt a lot about of finite element systems I feel I would better placed to both use and evaluate a greater range of potential choices. Its likely I would have therefore changed the finite element application and instead tried to use time I may have saved to improve the system for combining methods to add sophistication.

In the end I was glad that I selected C\# as the implementation language and would probably do so again with the exception of perhaps Java so as to have better cross platform compatibility. Initially I was also considering  Python although upon reflection I feel this would have been a mistake with implementation of the more object oriented aspects such as the element interface and subclass structure being made much more difficult by the language.



%what keep what change
