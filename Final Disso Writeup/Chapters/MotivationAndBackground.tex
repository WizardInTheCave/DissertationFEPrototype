
\section{Motivation and Background}
Over the past forty years FEA has emerged as a prominent technology for simulating complex real world engineering problems \cite{cite0, DolsakPaper94}. FEA works by solving a system of differential equations with each equation representing a single element in a geometric mesh. By doing this FEA is able to generate highly accurate approximations for the properties of complex physical systems \cite{DolsakPaper94} \cite{IntroductionToFE}. The method can also be highly computationally expensive with the complexity typically increasing exponentially with the model size \cite{DolsakPaper94}. Analysis therefore proves to be highly time consuming and costly for individuals and organisations to conduct \cite{ConsultRuleIntelltSystemFE}. \cite{cite03}\\


\subsection{Properties of finite element models}
Although this is a computer science as opposed to a mechanical engineering dissertation it's still important to briefly outline the general principals and properties that underpin the FE method in order to have a general appreciation for how the design and evaluation of the final software system was conducted. \\ 

\noindent
Finite Element Models have several key properties that need to be specified by the engineers who create them, the configuration of these properties greatly determine the results obtained from the models execution. The first of these properties that an FE model posesses is the mesh. The mesh is constructed out of nodes, points which act as intersections between the second componet- elements which are either a polygon or a polyhedron between the nodes. Nodes and elements are important concepts as they provide the theoretical framework for reasoning about the other properties of a mesh and hence the overall quality of the model \cite{IntroductionToFE}.\\ 

\noindent
In addition to the nodes and elements FE models also contain inputs and constraints. Inputs can be thought of as the phenomena from the outside world which is acting on the structure and consequently inducing some kind of physical effect on it. Inputs are used by engineers to model the conditions under which the structure will be expected to perform under when it is manufactured and enters operation. \\

% elements can be different types which have different types, typical types
% triangles and squares. 

\noindent
Constraints are another fundamental concept that describe where the model is attached to the outside world. When computing stresses induced through the model there needs to be an area through which the stresses are assumed to leave. FEA is only able to calculate the path of the stresses through the model and thus the overall stress at given points using the law of conservation of energy i.e. energy cannot be created or destroyed meaning that any energy provided to the system as input through for example force needs a means by which to leave it, the constraint. \\

\noindent
For example in figure 6 showing a suspension bridge model the simulation is to be run with the forces are induced upon the cables and the towers in the negative x direction as represented by the green vectors. The corresponding constraint area through which the force must leave is specified as the base of each bridge pillar and represented by multiple red arrows on each corresponding corner.\\

\noindent
The final piece of information needed in order to calculate the stresses through the model is material data. Material data is associated with the models elements and is usually defined using two main parameters which are:

\begin{itemize}
\item Youngs Modulus - The ratio of stress over strain for a given material, i.e. for an amount of internal force endured by a material how much does it deform, a material such as rubber therefore has a low value for Young's modulus while diamond has a high value\cite{YoungsModulus}
\item Poissons ratio - Amount of deformation that occurs perpendicular to the force that is applied to the material.
\end{itemize}

\noindent
For the sake of simplicity all structures used to evaluate the final software solution have assumed steel as their material property. Steel is a pretty common material used within manufacturing of many mechanical components and does not exhibit any abnormal properties. This is beneficial for evaluation as it removes variability in the results that could arise from selecting a more complex material.

\subsection{Limitations and general considerations}
\noindent
An important consideration when conducting FEA is the trade-off of a models accuracy against the time in which it can be solved. A major variable determining this trade-off is the models mesh structure which discretizes the problem so that simulation can be run on it. A mesh that is finer is more computationally expensive but also produces results of greater accuracy. It is therefore desirable to generate a mesh which is fine where accuracy is most needed but coarse where it is not \cite{cite04}. \\

\noindent
In every type of analysis that the FE method is used for (thermal, structural, fluid flow, electrical) there is a specific differential equation associated with each of the elements. In order to achieve overall convergence of the model the equations must be solved simultaneously to achieve a value for each of the discrete elements \cite{IntroductionToFE}. For this dissertation project attention will be given specifically to the problem of FEA meshing in the context of static structural analysis where the value calculated across each element is its stress. It makes sense to work on hybrid mesh refinement in the context of static structural analysis as it is likely the most common engineering application of the method and as such has the largest body of research that is relevant \cite{DolsakPaper94}\cite{IntroductionToFE}.\\

\noindent
For engineers the value obtained through computing the stresses under a particular set of conditions is feedback on the quality of their design. Ideally the results from an analysis will provide a good understanding of where the design is weak and how concentrated this weakness is. This information is used to either help verify the designs' quality or alternatively inform changes to its geometry or material properties so as to reduce stress on subsequent analysis \cite{cite06}.\\

\noindent
To understand the gradient of stress within a part of the model the mesh needs to be designed carefully. As each element can only display values calculated from its edge nodes a smooth gradient requires a higher concentration of elements in areas under higher stress. A high quality mesh is therefore considered to have a higher concentration of elements in areas of predicted high stress while retaining lower concentration elsewhere, thereby revealing weaknesses in the design while minimising the models runtime.\\

\noindent
Traditionally the automated mesh refinement process consists of computing stresses for a model with an initial coarse mesh and low computation cost, once rough stresses have been computed the elements in areas of higher stress can be divided recursively into additional elements in order to achieve smoother gradients on further executions \cite{cite03}. Figure 1 shows a mesh which has been refined in an area of higher stress thus providing a clearer indication of a components weakness. \\ 

\noindent
Unfortunately for many large models this method for refining a mesh is still excessively costly \cite{DolsakPaper91}. It therefore seems worth investigating use of alternative approaches posed by the field of computer science and artificial intelligence that could support the traditional high stress meshing approach. 

\begin{figure}
  \centerline{\includegraphics[width=110mm, scale=0.5]{../Graphics/stressedCorner.png}}
  \caption{Mesh refinement in corner under high stress image source: (\cite{HighStressCorner})}
  \label{fig:boat1}
\end{figure}