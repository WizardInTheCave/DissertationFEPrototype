\section{Introduction}

The overall aim of this project is to perform the task of refining a Finite Element Mesh (FEM), (see Motivation and Background for description of FEM) using a stress based method as is typical of FEM refinement in against an approach that uses techniques from Artificial Intelligence and Machine Learning. It should be possible to reason about the quality of a mesh produced using both methods analytically in order to evaluate the success of the approach and potential for continued use.\\

\noindent
Finite Element Analysis (FEA) is a method widely used across different engineering domains to simulate structures under certain conditions. As an engineer a typical iteration in the design process for mechanical components involves first designing the component within a CAD application before running a simulation on the design using knowledge of the conditions the is expected to perform under. Having defined the components geometry in continuous terms using CAD software a Finite Element System will discretize it into a representative mesh approximation before calculating values for each of the discrete elements when performing the simulation. This allows engineers to observe the effect the conditions have on the entire structure, see figure 1. \\ 

\noindent
The success of the project will be determined by implementation of a system that is able to combine the stress and AI based refinement methods in order to refine a mesh to of a quality comparable to that of a successful stress based method.

\noindent
This Dissertation starts with a review of the FEA process followed by looking at some of approaches people have taken to refine meshes automatically and gauge their quality before continuing to describe the design and discuss challenges of implementing the system. The system is then demonstrated as capable of correctly performing this task in the final evaluation section. 