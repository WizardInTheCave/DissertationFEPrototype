
\section{Further Work}
This section details some areas which given additional time to work on the project would at the very least be investigated, if not implemented. Each of these areas would hopefully provide some benefit in assisting to demonstrate the possibilities of hybrid methods.

\subsection{Gathering feedback from experienced engineers}
Approaching the end of the project it became clear that in order to better identify the systems strengths and weaknesses would require additional user testing by engineers who have experience conducting this type of analysis. Despite a lack of available time obtaining feedback from engineers with extensive applied industrial experience along with that of academics would have hopefully allowed for a more conclusive analysis of the systems and its ability to work across a greater number of general case scenarios. 

Part of the reason feedback for not obtaining user feedback was the difficulty of doing so given the available time available simply to implement the project and validate it for a selection of basic models. As such even if time had been available the ethical clearance required to collect user feedback at the start was not obtained.


\subsection{Improving usability through a web interface}
Although possible to visit various engineers in order to conduct feedback the process is both time consuming on my part and inconvenient for the participant as a rigid time for which to meet must be scheduled and a laptop containing the working software brought to them which they must design or transfer their model to before running it multiple times to obtain results. This scenario is at best inconvenient for the participants and pressures them into arriving at a conclusion within a relatively small time of experimenting with it. \\ 

\noindent
Instead by facilitating interaction with the system by means of a web interface the engineers would be able spend as little or as much time as they like experimenting with the system and allow them to submit feedback digitally allowing feedback to be obtained and aggregated from a much wider range of different sources separated by significant geographic distance. \\\ 

\noindent
To use the interface an engineer would simply need to submit a model they have already created  along with a JSON file containing edges they have designated as important for their model. LISA supports imports from multiple CAD formats including Standard for the Exchange of Product model data (STEP) and Initial Graphics Exchange Specification (IGES)) \cite{LISAManual}. Upon receiving the request the web server would the current project with their input data and having finished allow them to download the re meshed model along with the calculated stress data for analysis.


\section{Personal Reflections and Summary}
Comparing the current progress of the project against the plan things are on track to be completed as scheduled. So far I am pleased with both my research efforts and the functionality that I have been able to implement. My primary concern at present is that the ILP generated rules may not perform as well as expected when fully implemented or that they perform well but only for a limited set of geometries, which would be disappointing. While conducting research, development and implementation for the project I have worked methodically to best understand each problem as and when they occur and consider each of the possible solutions before committing to one. I feel this approached has saved me much time and has forced me to reach a better view about what exactly each subsystem should do and how to implement it. In hindsight if I were to repeat the first part of the project again I would have organised my time differently to spend less of it focussed on the details of coursework assignments for other modules into making further progress on the implementation of the rule system.


%Mention difficulty in initially predicting technbical challenges.

%Describe ambition behind project to challenge both comptuer science skills and improve knowledge of software methods for solving mechanical engineering problems.


%Things that went well
% - organised time well, made a good start
% - 

\newpage
\pagestyle{empty}
\begin{landscape}
\vspace*{1cm}
\hspace*{-3cm}
\includegraphics[width =700px, height=300px]{../Graphics/TimePlanUpdated2.png} \par
\hspace*{-1cm}

\end{landscape}

\newpage

\begin{changemargin}{\CMwidth}{\CMheight} 

\addcontentsline{toc}{section}{References}
\begin{thebibliography}{9}

\bibitem{cite0} Max D. Gunzburger, Janet S. Peterson \emph{Finite Element Methods} \url{https://people.sc.fsu.edu/~jburkardt/classes/fem\_2011/chapter1.pdf}

\bibitem{HandPRefinements} Adaptive Finite Element Techniques \url{http://www.cs.rpi.edu/~flaherje/pdf/fea8.pdf}

\bibitem{RRefinement} Scott McRae \emph{r-Refinement grid adaptation algorithms and issues}

\bibitem{DolsakPaper91} Bojan Dolsak and Anton Jezernik \emph{Mesh generation expert system for engineering analysis with FEM}

\bibitem{DolsakPaper94} Bojan Dolsak, Anton Jezernik \emph{A knowledge base for finite element mesh design} Artificial Intelligence in Engineering 9 (1994)

\bibitem{appOfILPToFEMeshDesign} Bojan Dolsak, Stephen Muggleton \emph{The Application of Inductive Logic Programming to Finite Element Mesh Design}

\bibitem{ConsultRuleIntelltSystemFE} Bojan Dolsak, Frank Reig, Reinhard Hackenschmidt \emph{Consultative Rule-Based Intelligent System for Finite Element Type Selection} Research Gate 2016

\bibitem{TraditionalHybridRefinement} Paul Dvorak \emph{Two meshing methods are better than one} \url{http://machinedesign.com/archive/two-meshing-methods-are-better-one}

\bibitem{NeuralNetworks} Larry Manevitz, Malik Yousef, Dan Givoli \emph{Automatic Mesh Generation (for Finite Element Method) Using Self-Organising Neural Networks}

\bibitem{caseBasedReasoning}Abid Ali Khan, Imran Ali Chaudhry2 \& Ali SaroshCase \emph{Case Based Reasoning Support for Adaptive Finite Element Analysis: Mesh Selection for an Integrated System}

\bibitem{MuggletonILP} Stephen Muggleton \emph{Inductive Logic Programming}

\bibitem{Golem} \url{http://www-ai.ijs.si/~ilpnet2/systems/golem.html}

\bibitem{ILPYoutubeLecture}Stephen Muggleton \emph{Logic based and Probabilistic Symbolic Learning} \url{https://www.youtube.com/watch?v=4CwdO5dWW98}

\bibitem{DittmerMeshQualityMet} Jeremy P. Dittmer, C. Greg Jensen, Michael Gottschalk, and Thomas Almy \emph{Mesh Optimisation Using a Genetic Algorithm to Control Mesh Creation Parameters}

\bibitem{PoorFEElementShapes} \url{http://danielpeter.github.io/rays.html}

\bibitem{cite03} Lina Vasiliauskiene, Romualdas BAUŠYS \emph{Intelligent Initial Finite Element Mesh Generation for Solutions of 2D Problems} INFORMATICA, 2002, Vol. 13, No. 2, 239–250 2002

\bibitem{cite04} E.Bellengera,Y.Benhafidb, N.Troussierb \emph{Framework for controlled cost and quality of assumptions in finite element analysis} Finite Elements in Analysis and Design 45 (2009) 25--36

\bibitem{IntroductionToFE} G. P. Nikishkov \emph{INTRODUCTION TO THE FINITE ELEMENT METHOD} \url{http://homepages.cae.wisc.edu/~suresh/ME964Website/M964Notes/Notes/introfem.pdf}

\bibitem{LISAManual} \url{http://www.lisafea.com/pdf/manual.pdf}

\bibitem{cite06}Nam-Ho Kim \emph{STRUCTURAL DESIGN USING FINITE ELEMENTS} http://web.mae.ufl.edu/nkim/eas6939/Opt\_FEM.pdf

\bibitem{cite07}\emph{Type of Finite Elements and Steps in FEA Process}\\
http://highered.mheducation.com/sites/dl/free/0073398144/934758/\\Ch07TypesOfFiniteElementsAndStepsInFEAProcess.pdf 

\bibitem{cantileverBeam} \url{https://www.quora.com/What-is-the-cantilever-beam-What-is-the-advantages-and-disadvantages-of-it}

\bibitem{LISAWebsite} \url{http://www.lisafea.com/purchase.html}

\bibitem{Doxygen} \url{http://www.stack.nl/~dimitri/doxygen/} 

\bibitem{ElementShapeQuality} \url{https://caeai.com/blog/will-poorly-shaped-elements-really-affect-my-solution}

\bibitem{AnsysCost} \url{http://mscnastrannovice.blogspot.co.uk/2013/04/how-much-does-ansys-cost.html}

\bibitem{HighStressCorner} \url{http://www.engineeringanalysisservices.com/moving-mesh-fea-analysis.php}

\bibitem{CSharpConvexHull} \url{http://loyc.net/2014/2d-convex-hull-in-cs.html}

\bibitem{YoungsModulus} \url{http://physicsnet.co.uk/a-level-physics-as-a2/materials/young-modulus/}

\bibitem{PossionsRatio} \url{http://silver.neep.wisc.edu/~lakes/PoissonIntro.html}

\bibitem{DelaunyTriangles} Jonathan Richard Shewchuk \emph{Delaunay Refinement Algorithms
for Triangular Mesh Generation} \url{https://people.eecs.berkeley.edu/~jrs/papers/2dj.pdf}



\end{thebibliography}
\appendix

\section{Element Types within LISA}
Here are shown the the visual specifications LISA provides for the ordering and layout of nodes for defining each type of element supported. Each of these types can be classified using the

\begin{figure}[!h]
\centering
\begin{subfigure}{.5\textwidth}
  \centering
  \includegraphics[width=0.3\linewidth]{../Graphics/LISA-quad4.png}
  \caption{quad4 element}
  \label{fig:sub1}
\end{subfigure}%
\begin{subfigure}{.5\textwidth}
  \centering
  \includegraphics[width=0.3\linewidth]{../Graphics/LISA-hex8.png}
  \caption{hex8 element}
  \label{fig:sub2}
\end{subfigure}
\label{fig:test}
\end{figure}


\begin{figure}[!h]
\centering
\begin{subfigure}{.5\textwidth}
  \centering
  \includegraphics[width=0.3\linewidth]{../Graphics/LISA-tri3.png}
  \caption{Specification for node ordering of tri3 element within LISA}
  \label{fig:sub1}
\end{subfigure}%
\begin{subfigure}{.5\textwidth}
  \centering
  \includegraphics[width=0.3\linewidth]{../Graphics/LISA-tet4.png}
  \caption{tet4 element}
  \label{fig:sub2}
\end{subfigure}
\label{fig:test}
\end{figure}


\begin{figure}[!h]
\centering
\begin{subfigure}{.5\textwidth}
  \centering
  \includegraphics[width=0.3\linewidth]{../Graphics/LISA-line2.png}
  \caption{line2 element}
  \label{fig:sub1}
\end{subfigure}%
\begin{subfigure}{.5\textwidth}
  \centering
  \includegraphics[width=0.3\linewidth]{../Graphics/LISA-line3.png}
  \caption{line3 element}
  \label{fig:sub2}
\end{subfigure}
\label{fig:test}
\end{figure}



\section{Input and output files}
Below can be seen the format of the input files for the system (A LISA .liml and a .json edge definition file


\begin{figure}[!h]
\centering
\begin{subfigure}{.5\textwidth}
  \centering
  \includegraphics[width=0.6\linewidth]{../Graphics/limlFileLayout.png}
  \caption{Cut down .liml file to show general content which largely defined the schema for the systems data model}
  \label{fig:sub1}
\end{subfigure}%
\begin{subfigure}{.5\textwidth}
  \centering
  \includegraphics[width=0.8\linewidth]{../Graphics/jsonEdgeFileLayout.png}
  \caption{A json file containing the edges of interest specified by an engineer, this is parsed and the rules are applied to determine the models meshing based on the input}
  \label{fig:sub2}
\end{subfigure}
\label{fig:test}
\end{figure}



\end{changemargin}
